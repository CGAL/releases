
\cleardoublepage
\chapter{Intersections}

\cgal\ provides functions to compute the intersection of any pair of
geometric objects. Many algorithms assume that the objects are in
general position, which is too restrictive in practice. When we
intersect two segments $s_1$ and $s_2$ we not only expect either no
intersection or an intersection in a point as a possible result, but
as well an intersection in a segment.  The latter case occurs, if
$s_1$ and $s_2$ are collinear and do overlap.

This leads to a two phase approach.  In the first phase, the type of
intersection is computed.  In the second phase the actual regions of
intersection can be retrieved, where the choice of the retrieval
function depends on the result of the first phase.

For example, if it is known after the first phase that two segments
intersect in a point, the point retrieval function should be called.
If they intersect in a segment, the segment retrieval function should
be called.

Before intersection queries between two objects can be done, the
objects must first be packed together in a pair.  For every pair of
classes there is a class that can hold a pair of objects of those
classes.  The purpose of such a pair class is twofold.  It holds an
enumeration type that specifies all the possible results.  Also, it
can maintain intermediate results from the first phase to the second
phase. Often, deciding what type of intersection takes place computes
a big part of the intersection result. It would be a waste to throw
this away.

In the following section we describe the steps in detail for the case
 of a segment line intersection. Other cases go similarly and are
 described more concisely afterwards.


\begin{classtemplate}{CGAL_Segment_2_Line_2_pair<R>}
\CCsection{Segment and Line}

\definition

An object \CCstyle{p} of type \classname\ holds two pointers to the
segment and the line that are to be intersected. The intersection
routines are member functions of this class.

The name of the pair class can be directly derived from the name of
the two object classes. The order in which the names appear in the
pair class name does not matter: always both names exist. So, in this
case, also \CCstyle{CGAL_Line_2_Segment_2_pair<R>} exists. This class has
exactly the same functionality, except that the order of the
arguments of the constructor should be reversed, reflecting the order
in which the objects appear in the name.


\creation
\creationvariable{p}

\constructor{CGAL_Segment_2_Line_2_pair(CGAL_Segment_2<R> *seg,
    CGAL_Line_2<R> *line)}{
introduces a variable \var\ of type \classname\ initialized with a pointer
to a segment and a pointer to a line.
As long as \var\ exists, both objects pointed to should not be
changed and should not go out of scope.
}

Class \classname\ defines an enumeration type that indicates what kind of
intersection occurs.
\begin{verbatim}
enum Intersection_results {NO, POINT, SEGMENT}
\end{verbatim}

Because this enumeration type belongs to the class it should be
qualified with the class name when it is used outside the class.  For
example, use \CCstyle{\classname ::POINT} to access the second
value.

Every intersection pair class has an enumeration type of this name.
The values differ from case to case, because the kind of
intersections that can occur is different for different objects. The
value \CCstyle{NO} is always present. It indicates that the two
objects do not intersect.

\operations
%\threecolumns{4.0cm}{3.5cm}

\method{CGAL_Segment_2_Line_2_pair<R>::Intersection_results intersection_type() const;}{
returns the type of intersection that takes place. The possible values are
described elsewhere. The enumeration type is declared in the class}

\method{void intersection(CGAL_Point_2 &point) const;}{
This method should only be called when the \CCstyle{intersection_type} method
returned \CCstyle{POINT}. It assigns the intersection point of the segment and
the line to the argument of the method.}

\method{void intersection(CGAL_Segment_2 &seg) const;}{
This method should only be called when the \CCstyle{intersection_type} method
returned \CCstyle{SEGMENT}.
The segment and the line intersect in a segment and this segment is assigned
to the argument of the method.}

\example

\begin{verbatim}
    CGAL_Segment_2<R> seg;
    CGAL_Line_2<R> line;

    CGAL_Line_2_Segment_2_pair<R> pair(&line, &seg);
    switch (pair.intersection_type()) {
    case CGAL_Line_2_Segment_2_pair<R>::NO:
        // handle the no intersection case.
        break;
    case CGAL_Line_2_Segment_2_pair<R>::POINT: {
        CGAL_Point_2<R> point;
        pair.intersection(point);
        // handle the point intersection case.
        break; }
    case CGAL_Line_2_Segment_2_pair<R>::SEGMENT: {
        CGAL_Segment_2<R> seg;
        pair.intersection(seg);
        // handle the segment intersection case.
        break; }
    }

\end{verbatim}

\end{classtemplate}

\section{Other Intersection Computations}

All intersection computations follow the same scheme as the segment line
intersection described in the previous paragraph.
\begin{itemize}

\item There is a pair class whose name can be derived from the names of
the object classes.
\item The pair class has an enumeration type called
\CCstyle{Intersection_results}holding the values that indicate
what kind of intersection takes place.
\item The pair class has a constructor taking two pointers to the objects that
are to be intersected.
\item There is a member function called \CCstyle{intersection_type} that
returns the kind of intersection.
\item There are several retrieval functions all called \CCstyle{intersection}
that take one argument.
\end{itemize}

The only thing that is different from pair to pair is the values of the
enumeration type and the retrieval functions. Those values are described here.
Because often values and retrieval functions occur in several intersection
pair classes, the presentation of the possible results is split in two stages.
First we describe all possible result values and what kind of retrieval
function goes with those values. Then we describe for all pairs which
enumeration values (and thus retrieval functions) apply.



\subsection{Possible Result Values}
\label{all_intersection_results}

Here we describe the values that can occur in a \CCstyle{Intersection_results}
enumeration type of a particular pair class.
With every enumeration value a retrieval function is associated. This
retrieval function, called \CCstyle{intersection}, takes one argument, which
is passed by reference. The precise type of the argument depends on the
dimensionality of the objects.

The following tables give the values of the enumeration type and the
parameter of the associated retrieval function. This should be read as
follows. A two dimensional pair class that has \CCstyle{SEGMENT} as one of the
values of \CCstyle{Intersection_results}, has a retrieval member function
\CCstyle{void intersection(CGAL_Segment_2<R> &seg)}.


\begin{table}[h]
\caption{Retrieval functions for two dimensional objects}
\begin{center}
\begin{tabular}{|l|l|}
\hline
return value        &    parameter type \\
\hline
\CCstyle{NO}        & no retrieval function \\
\CCstyle{POINT}     & \CCstyle{CGAL_Point_2} \\
\CCstyle{SEGMENT}   & \CCstyle{CGAL_Segment_2} \\
\CCstyle{RAY}       & \CCstyle{CGAL_Ray_2} \\
\CCstyle{LINE}      & \CCstyle{CGAL_Line_2} \\
\hline
\end{tabular}
\end{center}
\end{table}

\begin{table}[h]
\caption{Retrieval functions for three dimensional objects}
\begin{center}
\begin{tabular}{|l|l|}
\hline
return value        &    parameter type \\
\hline
\CCstyle{NO}        & no retrieval function \\
\CCstyle{POINT}     & \CCstyle{CGAL_Point_3} \\
\CCstyle{SEGMENT}   & \CCstyle{CGAL_Segment_3} \\
\CCstyle{RAY}       & \CCstyle{CGAL_Ray_3} \\
\CCstyle{LINE}      & \CCstyle{CGAL_Line_3} \\
\hline
\end{tabular}
\end{center}
\end{table}

If an inappropriate intersection retrieval function is called --- for
instance with a reference to a segment while \CCstyle{POINT} was
returned --- the result of the call is undefined. The error may be
detected and cause a core dump, some garbage may be returned or
anything else.

\subsection{All Possible Intersections}

Here follows a list of possible return types for pairs of objects that
are checked for intersection.  The order of the two objects is of no
importance. For every pair
\CCstyle{CGAL_Obj1_Obj2_pair<R>} there also exists
\CCstyle{CGAL_Obj2_Obj1_pair<R>}.

We leave out the result \CCstyle{NO} as possible
return value.


\begin{center}
\begin{tabular}{|l|l|l|}
\hline
type A      & type B                       &    enumeration values \\
\hline
\raisebox{-2.5mm}[0ex][0ex]{\CCstyle{CGAL_Line_2}} &
\raisebox{-2.5mm}[0ex][0ex]{\CCstyle{CGAL_Line_2}} &
\CCstyle{POINT} \\
 &  & \CCstyle{LINE} \\
\hline
\raisebox{-2.5mm}[0ex][0ex]{\CCstyle{CGAL_Segment_2}} &
\raisebox{-2.5mm}[0ex][0ex]{\CCstyle{CGAL_Line_2}} &
\CCstyle{POINT} \\
 &  & \CCstyle{SEGMENT} \\
\hline
\raisebox{-2.5mm}[0ex][0ex]{\CCstyle{CGAL_Segment_2}} &
\raisebox{-2.5mm}[0ex][0ex]{\CCstyle{CGAL_Segment_2}} &
\CCstyle{POINT} \\
 &  & \CCstyle{SEGMENT} \\
\hline
\raisebox{-2.5mm}[0ex][0ex]{\CCstyle{CGAL_Ray_2}} &
\raisebox{-2.5mm}[0ex][0ex]{\CCstyle{CGAL_Line_2}} &
\CCstyle{POINT} \\
 & & \CCstyle{RAY} \\
\hline
\raisebox{-2.5mm}[0ex][0ex]{\CCstyle{CGAL_Ray_2}} &
\raisebox{-2.5mm}[0ex][0ex]{\CCstyle{CGAL_Segment_2}} &
\CCstyle{POINT} \\
 &  & \CCstyle{SEGMENT} \\
\hline
 &  & \CCstyle{POINT} \\
\CCstyle{CGAL_Ray_2} & \CCstyle{CGAL_Ray_2} &
\CCstyle{SEGMENT} \\
 &  & \CCstyle{RAY} \\
\hline
\end{tabular}
\end{center}





