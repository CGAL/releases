\cleardoublepage
\chapter*{Preface}
%\chapter{Preface}


\cgal\ is the {\em Computational Geometry Algorithms Library} that is
developped by the {\sc Esprit} project \cgal\  which is carried out by
a consortium of seven sites: 
\Anchor{http://www.cs.ruu.nl}{Utrecht University} (The Netherlands), 
\Anchor{http://www.inf.ethz.ch/}{ETH Z\"urich} (Switzerland), 
\Anchor{http://www.inf.fu-berlin.de}{Free University of Berlin} (Germany), 
\Anchor{http://www.inria.fr/Equipes/PRISME-eng.html}{INRIA Sophia-Antipolis}
 (France),
\Anchor{http://www.mpi-sb.mpg.de}{Max-Planck Institute for Computer Science},
Saarbr\"ucken (Germany),
\Anchor{http://info.risc.uni-linz.ac.at}{RISC} Linz (Austria) and 
\Anchor{http://www.math.tau.ac.il/}{Tel-Aviv University} (Israel).
You find more information on the project on the 
\Anchor{http://www.cs.ruu.nl/CGAL}{\cgal\ home page}
\LatexHtml{at {\tt http://www.cs.ruu.nl/CGAL}}{}.

The library is written in \CC\ and consists of three parts: (1) the
{\em kernel}, (2) the {\em basic library}, and (3) {\em support libraries}.

\medskip
The \cgal\ kernel, more precisely its specification and partially
its implementation, is to some extent in imitation of its ancestors
in the \cgal\ consortium.
These ancestors are XYZ library \cite{XYZ}, developed at ETH Z\"urich, 
PlaGeo/SpaGeo \cite{PlaGeo}, developed at Utrecht University, 
{\sc \CC Gal} \cite{Protocgal}, developed at {\sc Inria} Sophia-Antipolis, France,
and the former geometric part of 
\Anchor{http://www.mpi-sb.mpg.de/LEDA/leda.html}{LEDA}~\cite{leda-manual}, a library for 
combinatorial and geometric computing, developed at MPI f\"ur 
Informatik, Saarbr\"ucken, Germany.
All people having contributed to one of these more or less 
proto-typical ancestors indirectly contributed to the \cgal\ kernel
as well.
Here, ``former geometric part of \leda'' means the geometry distributed
as part of \leda\ at the time when the idea of developing \cgal\ was born,
i.e.\ in 1994.
In the meantime \leda's geometry evolved considerably, and the 
development of \cgal\ and the evolution of \leda\ surely stimulated
each other.

\medskip
Among all the ancestors it is undoubtably \leda\ that has the most
success. It is hence not surprising that \cgal\ is designed such
that it will work with the combinatorial algorithm part of \leda\ in a
seamless way. As far as the geometric part of \leda\ is concerned
the long term goal of the \cgal\ project is to replace it.

In order to facilitate the life of a user of both
libraries we adopted the manual layout developed by the \leda\
project. In order to facilitate our life, the classes and algorithms
in \cgal\ make partially use of \leda, or are code that was
scavenged and adapted to our needs. Some of these features are
entirely hidden to the user and may be subject to change in
future releases, e.g., the memory management, others are visible, 
as for example the arithmetic and algebra packages, which support 
exact computation.


\medskip
The authors are grateful for the discussions and the feedback on Release~0.4 
of the kernel provided by the people in their local research groups.
We  would like to thank the \leda\ people for plenty of discussions
and for giving us liberate access to their code. We
especially would like to thank Michael Seel from MPI, who joined the
kernel desing group for a couple of weeks.
Furthermore we would like to thank Karsten 
Weihe and Dietmar K\"uhl from Konstanz University (Germany) who
gave us a valuable hint that helped solving a problem that occupied us
for a couple of weeks. 

\smallskip
We would like to extend our thanks to the institutes that gave us the
opportunity to work on the library before the Esprit project \cgal\
started.  {\sc Inria}, the Max-Planck Institute and Utrecht University
invested each more than a man year in the design and development of
the kernel.  The Free University of Berlin allowed two of us to
participate in the design of the kernel and to provide tools for
documentation. Each institute hosted the kernel design group for a
meeting or invited members of this group for a couple of days.
