% $Id: Ray_3.tex,v 1.2 1996/07/08 12:19:15 fabri Exp fabri $

\begin {classtemplate} {CGAL_Ray_3<R>}
\CCsection{3D Ray}

\definition
An object \CCstyle{r} of the data type \classname\ is a directed
straight ray in the three-dimensional Euclidean space $\E_3$. It starts
in a  point called the {\em source} of  \CCstyle{r} and it reaches infinity.


\creation
\creationvariable{r}

\CCstyle{#include <CGAL/Ray_3.h>}

\hidden \constructor{CGAL_Ray_3();}
             {introduces an uninitialized variable \var.}

\hidden \constructor{CGAL_Ray_3(const CGAL_Ray_3<R> &s);}
 	    {copy constructor.}

\constructor{CGAL_Ray_3(const CGAL_Point_3<R> &p, const Point_3 &q);}
            {introduces a ray \var\ 
             with source $p$ and passing through point $q$.}

\constructor{CGAL_Ray_3(const CGAL_Point_3<R> &p, const CGAL_Direction_3<R> &d)}
            {introduces a ray \var\ with source $p$ and with 
             direction $d$.}

\operations
\threecolumns{5cm}{4cm}

\hidden \method{CGAL_Ray_3<R> &operator=(const CGAL_Ray_3<R> &s);}
        {Assignment.}

\method{bool operator==(const CGAL_Ray_3<R> &h) const;}
       {Test for equality: two rays are equal, iff they have the same 
        source and the same direction.}

\method{bool operator!=(const CGAL_Ray_3<R> &h) const;}
       {Test for inequality.}



\method{CGAL_Point_3<R> source() const;}
       {returns the source of \var}

\method{CGAL_Point_3<R> point(int i) const;}
       {returns a point on \var. \CCstyle{point(0)} is the source.
        \CCstyle{point(i)}, with $i>0$, is different from the 
        source. \precond $i \geq 0$.}

\method{CGAL_Direction_3<R> direction() const;}
       {returns the direction of \var.}

\method{CGAL_Line_3<R>      supporting_line() const;}
       {returns the line supporting \var\ which has the same direction.}

\method{CGAL_Ray_3<R>       opposite() const;}
       {returns the ray with the same source and the opposite direction.}

\method{bool is_degenerate() const;}
       {ray \var\ is degenerate, if the source and the second defining
        point fall together (that is if the direction is degenerate).}


\method{bool has_on(const CGAL_Point_3<R> &p) const;}
       {A point is on \var, iff it is equal to the source 
        of \var, or if it is in the interior of \var.}
% 
% \method{bool collinear_has_on(const CGAL_Point_3<R> &p) const;}
%        {checks if point $p$ is on ray \var. This function is faster
%         than function \CCstyle{has_on()}.
%         \precond{$p$ is collinear to \var.}}
% 
\method{CGAL_Ray_3<R> transform(const CGAL_Aff_transformation_3<R> &t) const;}
       {returns the ray obtained by applying $t$ on the source 
        and on the direction of \var.}

\implementation

A ray is stored as a point and a direction.


\end {classtemplate} 

% $Log: Ray_3.tex,v $
% Revision 1.2  1996/07/08 12:19:15  fabri
% *** empty log message ***
%
% Revision 1.1  1996/03/13 15:42:07  fabri
% Initial revision
%
% Revision 1.1  1995/10/19 18:22:12  fabri
% Initial revision
%
