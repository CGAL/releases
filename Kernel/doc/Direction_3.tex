% $Id: Direction_3.tex,v 1.2 1996/07/08 12:19:15 fabri Exp fabri $

\begin {classtemplate} {CGAL_Direction_3<R>}
\CCsection{3D Direction}

\definition
An object of the class \classname\ is a vector in the three-dimensional 
vector space $\R^3$  where we forget about their length. They can be
viewed as unit vectors, although there is no normalization internally,
since this is error prone.  Directions are used whenever the length of
a vector does not matter, e.g.\ you can ask for the direction
orthogonal to an oriented plane, the direction of an oriented line,
and so on. 

\creation
\creationvariable{d}

\CCstyle{#include <CGAL/Direction_3.h>}

\hidden \constructor{CGAL_Direction_3();}
             {introduces an uninitialized direction \var.}

\hidden \constructor{CGAL_Direction_3(const CGAL_Direction_3<R> &d);}
 	    {copy constructor.}

\constructor{CGAL_Direction_3(const CGAL_Vector_3<R> &v);}
	    {introduces a direction \var\ passing through vector $v$.}

\constructor{CGAL_Direction_3(const R::RT &x, const R::RT &y);}
            {introduces a direction \var\ passing through the point 
             at $(x, y, z)$.}


\operations
\threecolumns{5cm}{4cm}

\hidden \method{CGAL_Direction_3<R> & operator=(const CGAL_Direction_3<R> &e);}
        {Assignment.}

\method{R::RT delta(int i) const;}
       {returns the i'th value of the slope of \var.
        \precond: $0 \leq i \leq 2$.}

\method{R::RT dx() const;}
       {returns the $dx$ value of the slope of \var.}

\method{R::RT dy() const;}
       {returns the $dy$ value of the slope of \var.}

\method{R::RT dz() const;}
       {returns the $dz$ value of the slope of \var.}


\method{bool operator==(const CGAL_Direction_3<R> &e) const;}
       {Test for equality.}

\method{bool operator!=(const CGAL_Direction_3<R> &e) const;}
       {Test for inequality.}


\method{CGAL_Direction_3<R>  operator-() const;}
       {The direction opposite to \var.}

\method{CGAL_Vector_3<R> vector() const;}
       {returns a vector that has the same direction as \var.}

\method{CGAL_Direction_3<R>  transform(const CGAL_Aff_transformation_3<R> &t) const;}
       {returns the direction obtained by applying $t$ on \var.}



\end {classtemplate} 

% $Log: Direction_3.tex,v $
% Revision 1.2  1996/07/08 12:19:15  fabri
% *** empty log message ***
%
% Revision 1.1  1996/03/13 15:42:07  fabri
% Initial revision
%
% Revision 1.1  1995/10/19 18:22:12  fabri
% Initial revision
%
