% $Id: Bitvector.tex,v 1.1 1996/03/13 15:42:07 fabri Exp fabri $

\cleardoublepage
\chapter{Bitvectors\label{Bitvectors}}

\begin {classtemplate} {CGAL_Bitvector<T>}
\CCsection{Bitvector}

\definition
An object of the class \classname\ is a container for boolean values.
The template can be instantiated with {\tt char}, {\tt short} or {\tt long}
as parameter. The number of bits that are provided depends on how many
bits your machine allocates for the built-in types. On a {\sc Sun} 
Sparc workstation it is 8, 16 and 32 bits, respectively. 

Important: To make your code portable \cgal\ provides classes with a digit
in the name: \CCstyle{CGAL_Bitvector8}, \CCstyle{CGAL_Bitvector16} and
\CCstyle{CGAL_Bitvector32}. You should use them. 


\creation
\creationvariable{b}

\constructor{CGAL_Bitvector();}
             {introduces a bitvector with all  bits initialized to `0'.}

\hidden \constructor{CGAL_Bitvector(const CGAL_Bitvector<T> &v);}
 	    {copy constructor.}

\constructor{CGAL_Bitvector(T t);}
            {introduces a bitvector \var\ initialized to $t$.}


\operations
\threecolumns{5cm}{4cm}

\method{bool  operator==(const CGAL_Bitvector<T> &q) const;}
       {Test for equality: Two bitvectors are equal, iff all corresponding 
        bits are equal.}

\method{bool operator!=(const CGAL_Bitvector<T> &q) const;}
       {Test for inequality.}

\method{int size();}
        {returns the size of the vector, that is the number of bits that 
         can be stored in \var.}

  
\method{void set_all();}
       {sets all bits to `1'.}

\method{void set_all();}
       {sets all bits to `0'.}

\method{T get_bit(int i) const;}
       {returns the i'th bit of \var.}

\method{T set_bit(int i);}
       {sets the i'th bit of \var to 1.}

\method{T reset_bit(int i);}
       {sets the i'th bit of \var to 0.}

\method{T invert_bit(int i);}
       {inverts the i'th bit of \var.}


\example

A bitvector is typically used when you design a class
with more than one boolean flag. As a \leda\ {\tt bool} 
is stored in a {\tt char} you save memory as soon as
you have more than one boolean values.

\begin{cprog}

  class Tree_node {
  public:
    bool red() {return b[0];}
    void set_red() { b.set_bit(0);}
    bool traversed() {return b[1];}
    void set_traversed() { b.set_bit(1);}
    
  private:
    CGAL_Bitvector<char> b;
    Tree_node *left, *right;
  };
\end{cprog} 

\end {classtemplate} 

% $Log: Bitvector.tex,v $
% Revision 1.1  1996/03/13 15:42:07  fabri
% Initial revision
%
% Revision 1.1  1995/10/19 18:22:12  fabri
% Initial revision
%
