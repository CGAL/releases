% $Id: Parabola_2.tex,v 1.3 1996/07/08 12:19:15 fabri Exp fabri $


\begin {classtemplate} {CGAL_Parabola_2<R>}
\CCsection{2D Parabola}

This class exist only for Cartesian representations.

\definition  
An object $par$ of the class \classname\ is a parabola in the
two-dimensional Euclidean plane $\E_2$. It is defined by a base point
$p$, a vector $v$ defining an angle of rotation and a curvature
$c$. The parabola is the set of points $q$, for which there is a
$\lambda$ with $q = p + \lambda \, v + \lambda^2 \, c \, v^\perp$, where
vector $v^\perp$ is orthogonal to $v$ and rotated in counterclockwise
direction.

We call the line to which the points of the parabola have
the same distance as to the focus the {\em director} of the parabola.
The director is directed such that the focus is on the same side of
the parabola and the line.

\creation
\creationvariable{par}

\CCstyle{#include <CGAL/Parabola_2.h>}

\hidden \constructor{CGAL_Parabola_2();}
             {introduces an uninitialized variable \var.}

\hidden \constructor{CGAL_Parabola_2(const CGAL_Parabola_2<R> &q);}
 	    {copy constructor.}


\def\CCalternateThreeColumn{\CCtrue}

\constructor{CGAL_Parabola_2(const CGAL_PointC2<FT> &p,
	                     const CGAL_Vector_2<R> &v,
	                     const R::RT &c);}
            {introduces a parabola \var\ with the base at $p$. 
             It is rotated around the base of the parabola by the angle 
             between the positive $x$-axis and the direction of vector $v$. 
             It has a curvature $c$.}

\constructor{CGAL_Parabola_2(const CGAL_Line_2<R> &l, const CGAL_Point_2<R> &p);}
	    {introduces a parabola \var\ which is the bisector between
             a line $l$ and a point $p$.}

\operations
\threecolumns{3.5cm}{4cm}

\hidden \method{CGAL_Parabola_2<R> & operator=(const CGAL_Parabola_2<R> &t2);}
        {Assignment.}

\method{bool operator==(const CGAL_Parabola_2<R> &p) const;}
       {Test for equality: two parabolas are equal, iff the equations
        are the same.}

\method{bool operator!=(const CGAL_Parabola_2<R> &p) const;}
       {Test for inequality.}



\method{bool is_degenerate() const;}
       {Parabola \var\ is degenerate, iff the curvature of \var\ is 0.}

\method{CGAL_Point_2<R> base() const;}
       {returns the base point of the parabola.}

\method{CGAL_Point_2<R> focus() const;}
       {returns the focus point of the parabola.}

\method{CGAL_Line_2<R> director() const;}
       {returns the director of \var.}

\method{CGAL_Vector_2<R> vector() const;}
       {returns the vector $v$ of the parametrized equation of \var.}

\method{R::RT curvature() const;}
       { returns the curvature of \var.}

\method{CGAL_Parabola_2<R> opposite() const;}
       {returns the parabola oriented in the opposite direction.}
	
\method{CGAL_Point_2<R> operator()(const R::FT &lambda) const;}
       {returns the point at \var.\CCstyle{base()} + $\lambda$ \var.\CCstyle{vector()}
        + $\lambda^2$ \, \var.\CCstyle{curvature()} \, \var.\CCstyle{vector()}$^\perp$.}

\method{int lambdas_at_x(const R::FT &x,
                         R::FT &lambda1,
                         R::FT &lambda2) const;}
       {computes parameters $\lambda_1$ and $\lambda_2$, for which 
        \var$(\lambda_1).x()==x$, \var$(\lambda_2).x()==x$. The return
        value is 0, 1 or 2 and it gives the number of  solutions.
        The value of $\lambda_i$  is undefined if the
        return value is smaller than $i$.}

\method{int lambdas_at_y(const R::FT &y,
                         R::FT &lambda1,
                         R::FT &lambda2) const;}
       {computes parameters $\lambda_1$ and $\lambda_2$, for which 
        \var$(\lambda_1).y()==y$, \var$(\lambda_2).y()==y$. The return
        value is 0, 1 or 2 and it gives the number of solutions.
        The value of $\lambda_i$  is undefined if the
        return value is smaller than $i$.}

\method{R::FT x_at_lambda(const R::FT &lambda) const;}
       {returns \var$(\lambda).x()$.}

\method{R::FT y_at_lambda(const R::FT &lambda) const;}
       {returns \var$(\lambda).y()$.}

\method{CGAL_Point_2<R> projection(const CGAL_Point_2<R> &p) const;}
       {returns the projection of $p$ on the parabola in the direction
        perpendicular to the director.}

\method{void projection(const CGAL_Point_2<R> &p, R::FT &lambda) const;}
       {as the function above. Additionally it assigns the value $\lambda$ to 
        the reference parameter \CCstyle{lambda},
        such that \var.$(\lambda)$ == \var.\CCstyle{projection(p)}.}

\method{CGAL_Oriented_side oriented_side(const CGAL_Point_2<R> &p) const;}
       {returns \CCstyle{CGAL_ON_POSITIVE_SIDE}, 
	\CCstyle{CGAL_ON_ORIENTED_BOUNDARY} or \CCstyle{CGAL_ON_NEGATIVE_SIDE},
        depending where point $p$ is relative to the oriented parabola \var.}


\method{bool has_on_positive_side(const CGAL_Point_2<R> &p) const;}
       {}

\method{bool has_on_negative_side(const CGAL_Point_2<R> &p) const;}
       {}

\method{bool has_on_boundary(const CGAL_Point_2<R> &p) const;}
       {}


\method{CGAL_Parabola_2<R>  transform(const CGAL_Aff_transformation_2<R> &t) const;}
       {returns the parabola obtained by applying $t$ on the 
        \CCstyle{director} and on the \CCstyle{focus} of \var.}



\end {classtemplate} 

% $Log: Parabola_2.tex,v $
% Revision 1.3  1996/07/08 12:19:15  fabri
% *** empty log message ***
%
% Revision 1.2  1996/03/13 15:42:07  fabri
% *** empty log message ***
%
% Revision 1.1  1996/02/22 14:09:59  fabri
% Initial revision
%
% Revision 1.1  1995/10/19 18:22:12  fabri
% Initial revision
%

