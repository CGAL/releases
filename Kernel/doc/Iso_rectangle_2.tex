% $Id: Iso_rectangle_2.tex,v 1.1 1995/10/19 18:22:12 fabri Exp fabri $

\begin {classtemplate} {CGAL_Iso_rectangle_2<R>}
\CCsection{2D Iso Rectangle}

\definition  An object $s$ of the data type \CCstyle{Iso_rectangle} is a
rectangle in the Euclidean plane $\E_2$ with sides parallel to the $x$ and
$y$ axis of the coordinate system.
 
Although they are represented in a canonical form by only two
vertices, namely the lower left and the upper right vertex, we provide
functions for ``accessing'' the other vertices as well. The vertices
are returned in counterclockwise order.

Iso-oriented rectangles and bounding boxes are quite similar. The
difference however is that bounding boxes have always double coordinates, 
whereas the coordinate type of an iso-oriented rectangle is chosen by
the user.


\creation
\creationvariable{r}

\CCstyle{#include <CGAL/Iso_rectangle_2.h>}

\hidden \constructor{CGAL_Iso_rectangle_2();}
             {introduces an uninitialized variable \var.}

\hidden \constructor{CGAL_Iso_rectangle_2(const CGAL_Iso_rectangle_2<R> &u);}
 	    {copy constructor.}

\constructor{CGAL_Iso_rectangle_2(const CGAL_Point_2<R> &p, 
	                          const CGAL_Point_2<R> &q);}
            {introduces an iso-oriented rectangle \var\ with diagonal
             opposite vertices $p$ and $q$. Note, that the object is 
             brought in the canonical form.}


\operations
\threecolumns{5cm}{4cm}

\hidden \method{CGAL_Iso_rectangle_2<R> & operator=(const CGAL_Iso_rectangle_2<R> &q);}
        {Assignment.}

\method{bool operator==(const CGAL_Iso_rectangle_2<R> &r2) const;}
       {Test for equality: two iso-oriented rectangles are equal, iff their
        lower left and their upper right vertices are equal.}

\method{bool operator!=(const CGAL_Iso_rectangle_2<R> &r2) const;}
       {Test for inequality.}


\method{CGAL_Point_2<R> vertex(int i) const;}
       {returns the i'th vertex modulo 4  of \var\ in counterclockwise order, 
        starting with the lower left vertex.}

\method{CGAL_Point_2<R> operator[](int i) const;}
       {returns  \CCstyle{vertex(i)}.}

\method{CGAL_Point_2<R> min() const;}
       {returns the lower left vertex of \var\ (= \CCstyle{vertex(0)}).}


\method{CGAL_Point_2<R> max() const;}
       {returns the upper right vertex of \var\ (= \CCstyle{vertex(2)}).}


\method{bool is_degenerate() const;}
       {the iso-oriented rectangle \var\ is degenerate, if all vertices
        are collinear.}


\method{CGAL_Bounded_side bounded_side(const CGAL_Point_2<R> &p) const;}
       {returns \CCstyle{CGAL_ON_BOUNDED_SIDE}, \CCstyle{CGAL_ON_BOUNDARY} or 
        \CCstyle{CGAL_ON_UNBOUNDED_SIDE}, depending on where point $p$ is.}

\method{bool has_on_boundary(const CGAL_Point_2<R> &p) const;}
       {}

\method{bool has_on_bounded_side(const CGAL_Point_2<R> &p) const;}
       {}

\method{bool has_on_unbounded_side(const CGAL_Point_2<R> &p) const;}
       {}

\method{CGAL_Bbox bbox() const;}
       {returns a bounding box containing \var. }

\method{CGAL_Iso_rectangle_2<R>  transform(const CGAL_Aff_transformation_2<R> &t) const;}
       {returns the iso-oriented rectangle obtained by applying $t$ on 
        the lower left and the upper right corner of \var.
        \precond The angle at a rotation must be a multiple of $\pi/2$,
        otherwise the resulting rectangle does not have the same side length.
        Note that rotating about an arbitrary angle can even result in
        a degenerate  iso-oriented rectangle.}





\end {classtemplate} 

% $Log: Iso_rectangle_2.tex,v $
% Revision 1.1  1995/10/19 18:22:12  fabri
% Initial revision
%
