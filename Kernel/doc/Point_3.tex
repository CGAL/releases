% $Id: Point_3.tex,v 1.2 1996/07/08 12:19:15 fabri Exp fabri $

\begin {classtemplate} {CGAL_Point_3<R>}
\CCsection{3D Point}

\definition
An object of the class \classname\ is a point in the three-dimensional
Euclidean space $\E_3$. 
%% 
%% \cgal\ defines a symbolic constant
%% \CCstyle{CGAL_ORIGIN}  which denotes the point at the origin. It can be used
%% wherever a point can be used, with the only exception that you can not
%% access its dimension as it is dimensionless.
%% 

Remember that \CCstyle{R::RT} and \CCstyle{R::FT} denote a ring type
and a field type. For the representation class
\CCstyle{CGAL_Cartesian<T>} the two types are equivalent. For the
representation class \CCstyle{CGAL_Homogeneous<T>} the ring type is
\CCstyle{R::RT} == \CCstyle{T} and the field type is \CCstyle{R::FT} == 
\CCstyle{CGAL_Quotient<T>}.

\creation
\creationvariable{p}

\CCstyle{#include <CGAL/Point_3.h>}

\hidden \constructor{CGAL_Point_3();}
             {introduces an uninitialized variable \var.}

\hidden \constructor{CGAL_Point_3(const CGAL_Point_3<R> &q);}
 	    {copy constructor.}

\constructor{CGAL_Point_3(const R::RT &hx, const R::RT &hy, const R::RT &hz, const R::RT &hw = R::RT(1));}
            {introduces a point \var\ initialized to $(hx/hw,hy/hw, hz/hw)$.
             If the third argument is not explicitely given it defaults
             to \CCstyle{R::RT(1)}.}


\operations
\threecolumns{5cm}{4cm}

\hidden \method{CGAL_Point_3<R> & operator=(const CGAL_Point_3<R> &q);}
        {Assignment.}

\method{bool operator==(const CGAL_Point_3<R> &q) const;}
       {Test for equality: Two points are equal, iff their $x$, $y$ and $z$
        coodinates are equal.}

\method{bool operator!=(const CGAL_Point_3<R> &q) const;}
       {Test for inequality.}



There are two sets of coordinate access functions, namely to the
homogeneous and to the Cartesian coordinates. They can be used
independently from the chosen representation type \CCstyle{R}.

\method{R::RT hx() const;}
       {returns the homogeneous $x$ coordinate.}

\method{R::RT hy() const;}
       {returns the homogeneous $y$ coordinate.}

\method{R::RT hz() const;}
       {returns the homogeneous $z$ coordinate.}

\method{R::RT hw() const;}
       {returns the homogenizing  coordinate.}

Here come the Cartesian access functions. Note that you do not loose
information with the homogeneous representation, because then the field
type is a quotient.

\method{R::FT x() const;}
       {returns the Cartesian $x$ coordinate, that is $hx/hw$.}

\method{R::FT y() const;}
       {returns the Cartesian $y$ coordinate, that is $hy/hw$.}

\method{R::FT z() const;}
       {returns the Cartesian $z$ coordinate, that is $hz/hw$.}


The following operations are for convenience and for making this
point class compatible with code for higher dimensional points.
Again they come in a Cartesian and homogeneous flavor.

\method{R::RT homogeneous(int i) const;}
       {returns the i'th homogeneous coordinate of \var, starting with 0.
        \precond $0\leq i \leq 3$.}

\method{R::FT cartesian(int i) const;}
       {returns the i'th Cartesian coordinate of \var, starting with 0.
        \precond $0\leq i \leq 2$.}

\method{R::FT operator[](int i) const;}
       {returns \CCstyle{cartesian(i)}.
        \precond $0\leq i \leq 2$.}

\method{int dimension() const;}
       {returns the dimension (the constant 3).}

\method{CGAL_Bbox_3 bbox() const;}
       {returns a bounding box containing \var.}

\method{CGAL_Point_3<R>  transform(const CGAL_Aff_transformation_3<R> &t) const;}
       {returns the point obtained by applying $t$ on \var.}



The following operations can be applied on points:

\function{CGAL_Vector_3<R> operator-(const CGAL_Point_3<R> &p,
                                    const CGAL_Point_3<R> &q);}
       {returns the difference vector between \CCstyle{q} and \CCstyle{p}.}

\function{CGAL_Point_3<R> operator+(const CGAL_Point_3<R> &p,
                                    const CGAL_Vector_3<R> &v);}
       {returns a point obtained by translating \CCstyle{p} by the 
        vector \CCstyle{v}.}

\function{CGAL_Point_3<R> operator-(const CGAL_Point_3<R> &p,
                                    const CGAL_Vector_3<R> &v);}
       {returns a point obtained by translating \CCstyle{p} by the 
        vector \CCstyle{-v}.}


\end {classtemplate} 
% $Log: Point_3.tex,v $
% Revision 1.2  1996/07/08 12:19:15  fabri
% *** empty log message ***
%
% Revision 1.1  1996/03/13 15:42:07  fabri
% Initial revision
%
% Revision 1.1  1995/10/19 18:22:12  fabri
% Initial revision
%
