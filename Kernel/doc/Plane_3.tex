% $Id: Plane_3.tex,v 1.2 1996/07/08 12:19:15 fabri Exp fabri $

\cleardoublepage

\chapter{3D Plane}

\begin {classtemplate} {CGAL_Plane_3<R>}
\CCsection{3D Plane}

\definition
An object \CCstyle{h} of the data type \classname\ is an oriented
plane in the three-dimensional Euclidean space $\E_3$. It is defined
by the set of points with coordinates  $(x,y,z)$ that satisfy the plane 
equation 

\begin{TexOnly}
\[{\cal H :}\;  a\, x +b\, y +c\, z + d = 0.\]
\end{TexOnly}
\begin{HtmlOnly}
H : a x + b y + c z + d = 0
\end{HtmlOnly}

\begin{TexOnly}%
The plane splits $\E_3$ in a {\em positive} and a {\em negative
side}%
.\footnote{See Section~\ref{Predicates} for the definition of \CCstyle{CGAL_Oriented_side}.}
\end{TexOnly}
\begin{HtmlOnly}
The plane splits <MATH><B>E</B><SUB>3</SUB></MATH> in a <EM> positive
</EM> and a <EM> negative side</EM>
(see CGAL_Oriented_side).
\end{HtmlOnly}
A point $p$ with coordinates $(px, py, pz)$ is on the positive side of
\CCstyle{h}, iff \LatexHtml{$a\, px +b\, py +c\, pz + d > 0$}{a px + b py +c pz + d > 0}, it is on the
negative side, iff \LatexHtml{$a\, px +b\, py\, +c < 0$}{a px + b py +c pz + d < 0}.



\creation
\creationvariable{h}

\CCstyle{#include <CGAL/Plane_3.h>}

\hidden \constructor{CGAL_Plane_3();}
             {introduces an uninitialized variable \var.}

\hidden \constructor{CGAL_Plane_3(const CGAL_Plane_3<R> &h);}
 	    {copy constructor.}

\constructor{CGAL_Plane_3(const R::RT &a, 
                          const R::RT &b,
                          const R::RT &c,
                          const R::RT &d)}
{introduces a plane \var\ defined by the equation
 \LatexHtml{$a\, px +b\, py +c\, pz + d = 0$}{a px + b py + c pz + d = 0}.}

\constructor{CGAL_Plane_3(const CGAL_Point_3<R> &p,
	                  const CGAL_Point_3<R> &q,
	                  const CGAL_Point_3<R> &r);}
{introduces a plane \var\ passing through the points \CCstyle{p},
 \CCstyle{q} and \CCstyle{r}. The plane is oriented such that \CCstyle{p}, 
 \CCstyle{q} and \CCstyle{r} are oriented in a positive sense 
 (that is counterclockwise) when seen from the positive side of \var.}


\constructor{CGAL_Plane_3(const CGAL_Point_3<R> &p,
	                  const CGAL_Direction_3<R>&d)}
{introduces a plane \var\ that passes through point \CCstyle{p} and
 that has as an orthogonal direction equal to \CCstyle{d}.}

\constructor{CGAL_Plane_3(const CGAL_Line_3<R> &l,
                          const CGAL_Point_3<R> &p)}
{introduces a plane \var\ that is defined through the  three points 
 \CCstyle{l.point(0)}, \CCstyle{l.point(1)} and \CCstyle{p}.}

\constructor{CGAL_Plane_3(const CGAL_Ray_3<R> &r,
                          const CGAL_Point_3<R> &p)}
{introduces a plane \var\ that is defined through the  three points 
 \CCstyle{r.point(0)}, \CCstyle{r.point(1)} and \CCstyle{p}.}

\constructor{CGAL_Plane_3(const CGAL_Segment_3<R> &s,
                          const CGAL_Point_3<R> &p)}
{introduces a plane \var\ that is defined through the  three points 
 \CCstyle{s.source()}, \CCstyle{s.target()} and \CCstyle{p}.}

\operations
\threecolumns{5cm}{4cm}

\hidden \method{CGAL_Plane_3<R> & operator=(const CGAL_Plane_3<R> &h);}
        {Assignment.}

\method{bool operator==(const CGAL_Plane_3<R> &h2) const;}
       {Test for equality: two planes are equal, iff they have a non 
        empty intersection and the same orientation.}

\method{bool operator!=(const CGAL_Plane_3<R> &h2) const;}
       {Test for inequality.}

\method{R::RT a() const;}
       {returns the first coefficient of \LatexHtml{${\cal H}$}{H}.}

\method{R::RT b() const;}
       {returns the second coefficient of \LatexHtml{${\cal H}$}{H}.}

\method{R::RT c() const;}
       {returns the third coefficient of \LatexHtml{${\cal H}$}{H}.}

\method{R::RT d() const;}
       {returns the fourth coefficient of \LatexHtml{${\cal H}$}{H}.}

\method{CGAL_Line_3<R> perpendicular_line(const CGAL_Point_3<R> &p) const;}
       {returns the line that is perpendicular to \var\ and that
        passes through point \CCstyle{p}. The line is oriented from
        the negative to the positive side of \var.}

\method{CGAL_Plane_3<R>      opposite() const;}
       {returns the plane with opposite orientation.}

\method{CGAL_Point_3<R>    projection(const CGAL_Point_3<R> &p) const;}
       {returns the vertical projection of $p$ on \var.}

\method{CGAL_Point_3<R> point() const;}
       {returns an arbitrary point on \var.}

\method{CGAL_Vector_3<R> orthogonal_vector() const;}
       {returns the vector that is orthogonal to \var\ and that
        is directed to the positive side of \var.}

\method{CGAL_Direction_3<R> orthogonal_direction() const;}
       {returns the direction that is orthogonal to \var\ and that
        is directed to the positive side of \var.}

\method{CGAL_Vector_3<R>      base1() const;}
       {returns a vector that is orthogonal to 
        \CCstyle{orthogonal_vector()}.}

\method{CGAL_Vector_3<R>      base2() const;}
       {returns a vector that is orthogonal to both 
        \CCstyle{orthogonal_vector()} and \CCstyle{base1()}, 
         such that 
        \CCstyle{CGAL_orientation( point(), point() + base1(), 
        point()+base2(), point() + orthogonal_vector() )} is positive.}

The following functions provide conversion between a plane and 
\cgal's two dimensional space. The transformation is affine, but
not necessarily an isometry. This means, the transformation preserves
combinatorics, but not distances.



\method{CGAL_Point_2<R>       to_2d(const CGAL_Point_3<R> &p) const;}
       {returns the image point of the projection of \CCstyle{p} 
       under an affine transformation, which maps \var\ onto the 
       $xy$-plane, with the $z$-coordinate removed.}

\method{CGAL_Point_3<R>       to_3d(const CGAL_Point_2<R> &p) const;}
       {returns a point $q$, such that \CCstyle{to_2d( to_3d( p ))}
        is equal to \CCstyle{p}.}


\method{CGAL_Oriented_side oriented_side(const CGAL_Point_3<R> &p) const;}
       {returns \CCstyle{CGAL_ON_POSITIVE_SIDE}, 
        \CCstyle{CGAL_ON_ORIENTED_BOUNDARY} or \CCstyle{CGAL_ON_NEGATIVE_SIDE},
        depending where point $p$ is relative to the oriented plane \var.
        }



For convenience we provide the following boolean functions:

\method{bool has_on(const  CGAL_Point_3<R> &p) const;}
       {}

\method{bool has_on_boundary(const  CGAL_Point_3<R> &p) const;}
       {}

\method{bool has_on_positive_side(const  CGAL_Point_3<R> &p) const;}
       {}

\method{bool has_on_negative_side(const  CGAL_Point_3<R> &p) const;}
       {}


\method{bool has_on(const  CGAL_Line_3<R> &l) const;}
       {}

\method{bool has_on_boundary(const  CGAL_Line_3<R> &l) const;}
       {}

\method{bool is_degenerate() const;}
       {Plane \var\ is degenerate, if the coefficients \CCstyle{a} and 
        \CCstyle{b} of the plane equation are zero.}

\method{CGAL_Plane_3<R>  transform(const CGAL_Aff_transformation_3<R> &t) const;}
       {returns the plane obtained by applying $t$ on a point of \var\ 
        and the orthogonal direction of \var.}

\end {classtemplate} 

% $Log: Plane_3.tex,v $
% Revision 1.2  1996/07/08 12:19:15  fabri
%
