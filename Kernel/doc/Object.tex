% $Id: Object.tex,v 1.3 1996/07/08 12:19:15 fabri Exp fabri $

\cleardoublepage
\chapter{A Generic Object} \label{Generic_Object}

Some functions can return different types of objects. A typical
\CC\ solution to this problem is to derive all possible return
types from a common base class, to return a pointer to this 
class and to perform a dynamic cast on this pointer. The class
presented in the following section provides an abstraction.

 
\begin{class} {CGAL_Object}
\CCsection{Object}

\definition  An object \CCstyle{obj} of the class \classname\ can
represent an arbitrary class. The only operations it provides is
to make copies and assignments, so that you can put them in lists
or arrays.

This class is not templated.


\creation
\creationvariable{obj}

\constructor{CGAL_Object();}
            {introduces an uninitialized variable.}

\constructor{CGAL_Object(const CGAL_Object &o);}
            {Copy constructor.}


\functiontemplate{T}{CGAL_Object CGAL_make_object(const T &t);}
{Creates an object that contains \CCstyle{t}.}

\threecolumns{5cm}{4cm}

\operations
\method{CGAL_Object &operator=(const CGAL_Object &o);}
            {Assignment.}


\function{bool CGAL_assign(CGAL_Class &c, const CGAL_Object &o);}
       {assigns \CCstyle{o} to \CCstyle{c} if \CCstyle{o}
        was constructed from an object of type \CCstyle{CGAL_Class}.
        Returns true, if the assigment was possible.}


\example

The object class is used as return value for the intersection computation,
as there are possibly different return values.

\begin{cprog}
{
  CGAL_Point_2< CGAL_Cartesian<double> > point;
  CGAL_Segment_2< CGAL_Cartesian<double> > segment,  segment_1, segment_2;

  cin >> segment_1 >> segment_2;

  CGAL_Object obj = CGAL_intersection(segment_1, segment_2);

  if (CGAL_assign(point, obj)) {
      /* do something with point */
  } else if ((CGAL_assign(segment, obj)) {
      /* do something with segment*/
  }
  /*  there was no intersection */
}
\end{cprog}

\medskip
The intersection routine itself looks roughly as follows:

\begin{cprog}

template < class R >
CGAL_Object  CGAL_intersection(CGAL_Segment_2<R> s1, CGAL_Segment_2<R> s2)
{
  if (/* intersection in a point */ ) {
     CGAL_Point_2<R> p = ... ;
     return CGAL_make_object(p);
  } else if (/* intersection in a segment */ ) {
     GAL_Segment_2<R> s = ... ;
     return CGAL_make_object(s);
  }
  return CGAL_Object();
}
\end{cprog} 
\end{class} 

% $Log: Bbox_2.tex,v $
