% $Id: Line_3.tex,v 1.2 1996/07/08 12:19:15 fabri Exp fabri $

\begin {classtemplate} {CGAL_Line_3<R>}
\CCsection{3D Line}

\definition
An object \CCstyle{l} of the data type \classname\ is a directed
straight line in the three-dimensional Euclidean space $\E_3$.

\creation
\creationvariable{l}

\CCstyle{#include <CGAL/Line_3.h>}

\hidden \constructor{CGAL_Line_3();}
             {introduces an uninitialized variable \var.}

\hidden \constructor{CGAL_Line_3(const CGAL_Line_3<R> &h);}
 	    {copy constructor.}

\constructor{CGAL_Line_3(const CGAL_Point_3<R> &p, const CGAL_Point_3<R> &q);}
            {introduces a line \var\ passing through the points $p$ and $q$. 
             Line \var\ is directed from $p$ to $q$.}


\constructor{CGAL_Line_3(const CGAL_Point_3<R> &p, const CGAL_Direction_3<R>&d)}
            {introduces a line \var\ passing through point $p$ with 
             direction $d$.}

\operations
\threecolumns{5cm}{4cm}

\hidden \method{CGAL_Line_3<R> & operator=(const CGAL_Line_3<R> &h);}
        {Assignment.}

\method{bool operator==(const CGAL_Line_3<R> &h) const;}
       {Test for equality: two lines are equal, iff they have a non 
        empty intersection and the same direction.}

\method{bool operator!=(const CGAL_Line_3<R> &h) const;}
       {Test for inequality.}

\method{CGAL_Plane_3<R>      perpendicular_plane(const CGAL_Point_3<R> &p) const;}
       {returns the plane perpendicular to \var\ passing through $p$.}

\method{CGAL_Line_3<R>      opposite() const;}
       {returns the line with opposite direction.}

\method{CGAL_Point_3<R>    projection(const CGAL_Point_3<R> &p) const;}
       {returns the vertical projection of $p$ on \var.}

\method{CGAL_Point_3<R> point(int i) const;}
       {returns an arbitrary point on \var. It holds 
        \CCstyle{point(i) == point(j)}, iff \CCstyle{i==j}.}

\method{CGAL_Direction_3<R> direction() const;}
       {returns the direction of \var.}

\method{bool is_degenerate() const;}
       {line \var\ is degenerate, if the coefficients \CCstyle{a} and 
        \CCstyle{b} of the line equation are zero.}


\method{bool has_on(const CGAL_Point_3<R> &p) const;}
       {}

\method{CGAL_Line_3<R>  transform(const CGAL_Aff_transformation_3<R> &t) const;}
       {returns the line obtained by applying $t$ on a point on \var\ 
        and the direction of \var.}




\end {classtemplate} 

% $Log: Line_3.tex,v $
% Revision 1.2  1996/07/08 12:19:15  fabri
% *** empty log message ***
%
% Revision 1.1  1996/03/13 15:42:07  fabri
% Initial revision
%
% Revision 1.1  1995/10/19 18:22:12  fabri
% Initial revision
%
