\clearpage
\begin {classtemplate} {CGAL_Quotient<NT>}
\label{CGAL_Quotient}
\CCsection{Quotient}

\definition
An object of the class \CCstyle{CGAL_Quotient<NT>} is an element of the 
ring of quotients of the integral domain type \CCstyle{NT}.
If \CCstyle{NT} behaves like an integer, \CCstyle{CGAL_Quotient<NT>}
behaves like the rational numbers. \leda's class
\CCstyle{rational} has been the basis for \CCstyle{CGAL_Quotient<NT>}.
A \CCstyle{CGAL_Quotient<NT>} \CCstyle{q} is represented as a pair of 
\CCstyle{NT}s, representing numerator and denominator.


\creation
\creationvariable{q}

\constructor{CGAL_Quotient();}
             {introduces an uninitialized variable \var.}

\hidden \constructor{CGAL_Quotient(const CGAL_Quotient<NT> &q);}
 	    {copy constructor.}

\constructor{CGAL_Quotient(const NT& n)}
            {introduces the quotient \CCstyle{n/1}.}

\constructor{CGAL_Quotient(const NT& n, const NT& d)}
            {introduces the quotient \CCstyle{n/d}.}


\operations
The arithmetic operations $+,\ -,\ *,\ /,\ +=,\
-=,\ *=,\ /=,\ -$(unary),
the comparison operations $<,\ <=,\ >,\ 
>=,\ ==,\ !=$ and the stream operations are all available.

\threecolumns{5cm}{4cm}

There are two access functions, namely to the
numerator and the denominator of a quotient.
Note that these values are not uniquely defined. 
It is guaranteed that \CCstyle{q.numerator()} and 
\CCstyle{q.denominator()} return values \CCstyle{nt_num} and
\CCstyle{nt_den} such that \CCstyle{q = nt_num/nt_den}, only
if  \CCstyle{q.numerator()} and \CCstyle{q.denominator()} are called
consequtively wrt \CCstyle{q}, i.e.~\CCstyle{q} is not involved in 
any other operation between these calls.

\method{NT numerator() const;}
       {returns a numerator of \CCstyle{q}.}

\method{NT denominator() const;}
       {returns a denominator of \CCstyle{q}.}

\hidden \method{CGAL_Quotient<NT>& normalize();}
{}

The following functions are added to fulfill the \cgal\ requirements
on number types.

\function{double CGAL_to_double(const CGAL_Quotient<NT>& q);}
       {returns some double approximation to \CCstyle{q}.}

\function{bool  CGAL_is_valid(const CGAL_Quotient<NT>& q);}
       {returns true, if numerator and denominator are valid.}

\function{bool  CGAL_is_finite(const CGAL_Quotient<NT>& q);}
       {returns true, if numerator and denominator are finite.}

\end {classtemplate} 

