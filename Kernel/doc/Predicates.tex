
\cleardoublepage
\chapter{Geometric Predicates} \label{Predicates}

\section{Ordertypes}


\threecolumns{3cm}{5cm}

In geometric algorithms we often want to know the enumeration type 
of  a sequence of $d$ points in $d-1$ dimensional space. \cgal\
provides the following enumeration type:


\globalenum{enum  CGAL_Sign { CGAL_NEGATIVE   = -1,
                              CGAL_ZERO,
                              CGAL_POSITIVE
                            };}

and a \CCstyle{typedef CGAL_Sign, CGAL_Orientation}. For the
two-dimensional space, different names are often used in the
literature. Here one wants to know whether three points perform a {\em
leftturn}, or a {\em rightturn}, or if they are {\em collinear}. The
latter includes the case that two or even all three points have the
same coordinates. Therefor \cgal\ provides also

\globalenum{enum  CGAL_Orientation { CGAL_CLOCKWISE   = -1,
                                   CGAL_COLLINEAR,
                                   CGAL_COUNTERCLOCKWISE };}


\globalvariable{const CGAL_Orientation CGAL_LEFTTURN  = CGAL_POSITIVE;}
\globalvariable{const CGAL_Orientation CGAL_RIGHTTURN = CGAL_NEGATIVE;}

\globalvariable{const CGAL_Orientation CGAL_COUNTERCLOCKWISE = CGAL_POSITIVE;}
\globalvariable{const CGAL_Orientation CGAL_CLOCKWISE = CGAL_NEGATIVE;}

\globalvariable{const CGAL_Orientation CGAL_COPLANAR  = CGAL_ZERO;}
\globalvariable{const CGAL_Orientation CGAL_COLLINEAR = CGAL_ZERO;}
\globalvariable{const CGAL_Orientation CGAL_DEGENERATE = CGAL_ZERO;}




\function{CGAL_Orientation CGAL_orientation(const CGAL_Point_2<R> &p,
                                   const CGAL_Point_2<R>&q,
                                   const CGAL_Point_2<R> &r);}
{returns \CCstyle{CGAL_LEFTTURN}, if $r$ lies to the left of the oriented 
line $l$ defined by $p$ and $q$, returns \CCstyle{CGAL_RIGHTTURN} if $r$ 
lies to the right of $l$ and returns \CCstyle{CGAL_COLLINEAR} if $r$ lies
on $l$.}

\pagebreak[3]
The following functions  return \CCstyle{true} or \CCstyle{false}
depending on whether the orientation of three points is equal to
one of the before mentioned constants.


\function{bool CGAL_leftturn(const CGAL_Point_2<R> &p,
                             const CGAL_Point_2<R> &q,
                             const CGAL_Point_2<R> &r);}{}

\function{bool CGAL_rightturn(const CGAL_Point_2<R> &p,
                              const CGAL_Point_2<R> &q,
                              const CGAL_Point_2<R> &r);}{}

\function{bool CGAL_collinear(const CGAL_Point_2<R> &p, 
                              const CGAL_Point_2<R> &q, 
                              const CGAL_Point_2<R> &r);}{}

Finally, there are some special cases of collinearity. If we not only
want to know if three points are collinear but also if they respect a 
certain order on the line, this is the function to call:


\function{bool CGAL_between(const CGAL_Point_2<R> &p, 
                            const CGAL_Point_2<R> &q, 
                            const CGAL_Point_2<R> &r);}
         {returns \CCstyle{true}, iff the three points are collinear and 
          \CCstyle{q} lies between \CCstyle{p} and \CCstyle{r}.}


If we already know that three points are collinear, we should not
check it again (as it is an expensive operation). 

\function{bool CGAL_collinear_between(const CGAL_Point_2<R> &p,
                                      const CGAL_Point_2<R> &q,
                                      const CGAL_Point_2<R> &r);}
         {returns \CCstyle{true}, iff \CCstyle{q} lies between \CCstyle{p} 
          and \CCstyle{r}. \precond \CCstyle{p, q} and \CCstyle{r} are 
	collinear.}



\subsubsection*{Implementation}

 The orientation computation boils down to a computation of the sign
of a determinant. For some number types this computation can be done
exact. For some numbertypes there is a direct way of computing the sign 
of a determinant without computing the determinant itself~\cite{ABDPY}. 
Moreover, this approach is numerically more stable and efficient.


\section{Relative Position}

Geometric objects in \cgal\ have member functions that test the
position of a point relative to the object.  Full dimensional objects
and their boundaries are represented by the same type, e.g.\
halfspaces and hyperplanes are not distinghuished, neither are spheres
and circles. Such objects split the ambient space into two
full-dimensional parts, a bounded part and an unbounded part (e.g.\
circles), or two unbounded parts (e.g.\ hyperplanes). By default these
objects are oriented, i.e., one of the resulting parts is called the
positive side, the other one is called the negative side. Both of
these may be unbounded.

For these objects there is a function \CCstyle{oriented_side()} that
determines whether a test point is on the positive side, the negative
side, or on the oriented boundary. These function returns an enumeration
type


\globalenum{enum  CGAL_Oriented_side
                  { CGAL_ON_NEGATIVE_SIDE = -1,
                    CGAL_ON_ORIENTED_BOUNDARY,
                    CGAL_POSITIVE_SIDE
                  }; }

Accordingly there are member functions
\CCstyle{has_on_positive_side(CGAL_Point_2<R>)},
\CCstyle{has_on_boundary(CGAL_Point_2<R>)} and
\CCstyle{has_on_negative_side(CGAL_Point_2<R>)} returning a boolean value.

Those objects that split the space in a bounded and an unbounded part, hava
a member function \CCstyle{bounded_side()} with the return type


\globalenum{enum  CGAL_Bounded_side
                  { CGAL_ON_BOUNDED_SIDE = -1,
                    CGAL_ON_BOUNDARY,
                    CGAL_ON_UNBOUNDED_SIDE
                  }; }

Accordingly there are member functions
\CCstyle{has_on_bounded_side(CGAL_Point_2<R>)},
\CCstyle{has_on_unbounded_side(CGAL_Point_2<R>)} returning a boolean value.

If an object is lower dimensional, e.g.\ a triangle in three-dimensional
space or a segment in tw-dimensional space, there is only a test whether a
point belongs to the object or not. This member function, which takes a 
point as an argument and returns a boolean value, is called \CCstyle{has_on()}.


\section{The Incircle Test}

Instead of constructing a circle and performing the test if a 
given point lies inside or outside you might use the following
predicate:

\function{CGAL_Bounded_side CGAL_side_of_bounded_circle(
	                           const CGAL_Point_2<R> &p, 
	                           const CGAL_Point_2<R> &q,
                                   const CGAL_Point_2<R> &r, 
                                   const CGAL_Point_2<R> &test);}
         {returns the relative position of point \CCstyle{test}
          to the circle defined by $p$, $q$ and $r$. The order
          of the points $p$, $q$ and $r$ does not matter.
          \precond \CCstyle{p, q} and \CCstyle{r} are not collinear.}
 
\function{CGAL_Oriented_side CGAL_side_of_oriented_circle(
	                           const CGAL_Point_2<R> &p, 
	                           const CGAL_Point_2<R> &q,
                                   const CGAL_Point_2<R> &r, 
                                   const CGAL_Point_2<R> &test);}
         {returns the relative position of point \CCstyle{test}
          to the oriented circle defined by $p$, $q$ and $r$.
	  The order of the points $p$, $q$ and $r$ is important,
	  since it determines the orientation of the implicitely
          constructed circle.
          \precond \CCstyle{p, q} and \CCstyle{r} are not collinear.}


\section{Comparison Results}

\globalenum{enum  CGAL_Comparison_result { CGAL_SMALLER   = -1,
                                           CGAL_EQUAL,
                                           CGAL_LARGER
                                 	};}


\section{Comparison of Coordinates of Points}

In order to check if two points have the same $x$ or $y$ coordinate
we provide the following functions. They allow to write code that
does not depend on the representation type.

\CCstyle{#include <CGAL/predicates_on_points_2.h>}

\function{bool CGAL_x_equal(const CGAL_Point_2<R> &p,
                            const CGAL_Point_2<R> &q);}
         {returns \CCstyle{true}, iff \CCstyle{p} and \CCstyle{q}
	  have the same \CCstyle{x}-coordinate.}

\function{bool CGAL_y_equal(const CGAL_Point_2<R> &p,
                            const CGAL_Point_2<R> &q);}
         {returns \CCstyle{true}, iff \CCstyle{p} and \CCstyle{q}
	  have the same \CCstyle{y}-coordinate.}


The above functions are decision versions of the following 
comparison functions returning a \CCstyle{CGAL_Comparison_result}.

\function{CGAL_Comparison_result CGAL_compare_x(const CGAL_Point_2<R> &p,
                            	                const CGAL_Point_2<R> &q);}
	{}

\function{CGAL_Comparison_result CGAL_compare_y(const CGAL_Point_2<R> &p,
                            	                const CGAL_Point_2<R> &q);}
	{}


\cgal\ offers the same functions for points given implicitely as intersection
of two lines.  We provide these functions because we can provide
better code for the test, than simply computing the intersection and
calling the respective function for points.

\precond Lines that define an intersection point may not be parallel.

\CCstyle{#include <CGAL/predicates_on_lines_2.h>}



\function{CGAL_Comparison_result CGAL_compare_x(const CGAL_Point_2<R> &p,
                                                const CGAL_Line_2<R> &l1,
                                                const CGAL_Line_2<R> &l2);}
	{compares the $x$-coordinates of $p$ and the intersection 
         of lines $l1$ and $l2$ (Figure~\ref{fig-compare} (a)).}


\function{CGAL_Comparison_result CGAL_compare_x(const CGAL_Line_2<R> &l,
                                                const CGAL_Line_2<R> &h1,
                                                const CGAL_Line_2<R> &h2);}
	{compares the $x$-coordinates of  the intersection of line $l$
         with line $h1$ and with line $h2$ (Figure~\ref{fig-compare} (b)).}


\function{CGAL_Comparison_result CGAL_compare_x(const CGAL_Line_2<R> &l1,
                                                const CGAL_Line_2<R> &l2,
                                                const CGAL_Line_2<R> &h1,
                                                const CGAL_Line_2<R> &h2);}
	{compares the $x$-coordinates of the intersection of lines $l1$
         and $l2$ and  the intersection of lines $h1$ and $h2$
	 (Figure~\ref{fig-compare} (c)).}
	

\function{CGAL_Comparison_result CGAL_compare_y(const CGAL_Point_2<R> &p,
                                                const CGAL_Line_2<R> &l1,
                                                const CGAL_Line_2<R> &l2);}
	{compares the $y$-coordinates of $p$ and the intersection of lines
         $l1$ and $l2$ (Figure~\ref{fig-compare} (a)).}


\function{CGAL_Comparison_result CGAL_compare_y(const CGAL_Line_2<R> &l,
                                                const CGAL_Line_2<R> &h1,
                                                const CGAL_Line_2<R> &h2);}
	{compares the $y$-coordinates of the intersection of line $l$
         with line $h1$ and with line $h2$. (Figure~\ref{fig-compare} (b))}


\function{CGAL_Comparison_result CGAL_compare_y(const CGAL_Line_2<R> &l1,
                                                const CGAL_Line_2<R> &l2,
                                                const CGAL_Line_2<R> &h1,
                                                const CGAL_Line_2<R> &h2);}
	{compares the $y$-coordinates of the intersection of lines $l1$
         and $l2$ and  the intersection of lines $h1$ and $h2$ 
	(Figure~\ref{fig-compare} (c))}


\begin{figure}[h]
\centerline{\Ipe{compare.ipe}}
\caption{Comparison of the $x$ or $y$-coordinates of the (implicitely
given) points in the boxes.\label{fig-compare}}
\end{figure} 


The following functions compare the \CCstyle{y} coodinate of an point
(that may be given implicitely) with a line. 

\precond If the point is given as an intersection of two lines these
lines may not be parallel. Lines where points are projected on may not be
vertical.

\function{CGAL_Comparison_result CGAL_compare_y_at_x(const CGAL_Point_2<R> &p,
                                           	const CGAL_Line_2<R> &h);}
        {compares the $y$-coordinates of $p$ and the vertical projection
         of \CCstyle{p} on \CCstyle{h} (Figure~\ref{fig-compare2} (d)).}


\function{CGAL_Comparison_result CGAL_compare_y_at_x(const CGAL_Point_2<R> &p,
                                           const CGAL_Line_2<R> &h1,
                                           const CGAL_Line_2<R> &h2);}
{This function compares the $y$-coordinates of the vertical projection 
 of \CCstyle{p} on \CCstyle{h1} and on \CCstyle{h2} (Figure~\ref{fig-compare2} (e)).}

\function{CGAL_Comparison_result CGAL_compare_y_at_x(const CGAL_Line_2<R> &l1,
                                           const CGAL_Line_2<R> &l2,
                                           const CGAL_Line_2<R> &h);}
      {Let $p$ be the intersection of lines $l1$ and $l2$.
       This function compares the $y$-coordinates of $p$ and 
       the vertical projection of \CCstyle{p} on \CCstyle{h}
	 (Figure~\ref{fig-compare2} (f))}


\function{CGAL_Comparison_result CGAL_compare_y_at_x(const CGAL_Line_2<R> &l1,
                                           const CGAL_Line_2<R> &l2,
                                           const CGAL_Line_2<R> &h1,
                                           const CGAL_Line_2<R> &h2);}
{Let $p$ be the intersection of lines $l1$ and $l2$. This function 
 compares the $y$-coordinates of the vertical projection of \CCstyle{p} on 
 \CCstyle{h1} and on \CCstyle{h2} (Figure~\ref{fig-compare2} (g)).}

\begin{figure}[h]
\centerline{\Ipe{compare2.ipe}}
\caption{Comparison of the $y$-coordinates of the (implicitely given)
         points in the boxes, at an $x$-coordinate. The $x$-coordinate
         is either given explicitely (disc) or implicitely (circle).
	 \label{fig-compare2}}
\end{figure} 


For lexicographical comparison \cgal\ provides

\function{CGAL_Comparison_result
CGAL_compare_lexicographically_xy(const CGAL_Point_2<R>& p,
                                  const CGAL_Point_2<R>& q);}
      {Compares the Cartesian coordinates of points \CCstyle{p} and
       \CCstyle{q} lexicographically in $xy$ order: first 
       $x$-coordinates are compared, if they are equal, $y$-coordinates
       are compared.}

In addition, \cgal\ provides the following comparison functions
returning \CCstyle{true} or \CCstyle{false} depending on the result
of \CCstyle{CGAL_compare_lexicographically_xy(p,q)}.

\function{bool 
CGAL_lexicographically_xy_smaller_or_equal(const CGAL_Point_2<R>& p,
                                           const CGAL_Point_2<R>& q);} {}

\function{bool
CGAL_lexicographically_xy_smaller(const CGAL_Point_2<R>& p,
                                  const CGAL_Point_2<R>& q);} {}

\function{bool
CGAL_lexicographically_xy_larger_or_equal(const CGAL_Point_2<R>& p,
                                          const CGAL_Point_2<R>& q);} {}

\function{bool
CGAL_lexicographically_xy_larger(const CGAL_Point_2<R>& p,
                                 const CGAL_Point_2<R>& q);} {}
