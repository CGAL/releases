% $Id: Triangle_2.tex,v 1.3 1996/07/08 12:19:15 fabri Exp fabri $


\begin {classtemplate} {CGAL_Triangle_2<R>}
\CCsection{2D Triangle}

\definition  An object $t$ of the class \classname\ is a triangle 
in the two-dimensional Euclidean plane~$\E_2$. 
Triangle  $t$  is oriented, i.e., its boundary has
clockwise or counterclockwise orientation. We call the side to the left
of the boundary the positive side and the side to the right of the
boundary the negative side.

As any Jordan curve the boundary of a triangle splits the plane in
two open regions, a bounded one and an unbounded one. 

\creation
\creationvariable{t}

\CCstyle{#include <CGAL/Triangle_2.h>}

\hidden \constructor{CGAL_Triangle_2();}
             {introduces an uninitialized variable \var.}

\hidden \constructor{CGAL_Triangle_2(const CGAL_Triangle_2<R> &u);}
 	    {copy constructor.}


\def\CCalternateThreeColumn{\CCtrue}
\constructor{CGAL_Triangle_2(const CGAL_Point_2<R> &p, 
	                     const CGAL_Point_2<R> &q, 
	                     const CGAL_Point_2<R> &r);}
            {introduces a triangle \var\ with vertices $p$,  $q$ and $r$.}


\operations
\threecolumns{4.5cm}{4cm}

\hidden \method{CGAL_Triangle_2<R> & operator=(const CGAL_Triangle_2<R> &t2);}
        {Assignment.}

\method{bool operator==(const CGAL_Triangle_2<R> &t2) const;}
       {Test for equality: two triangles are equal, iff there exists a 
        cyclic permutation of the vertices of $t2$, such that they are 
        equal to the vertices of \var.}

\method{bool operator!=(const CGAL_Triangle_2<R> &t2) const;}
       {Test for inequality.}



\method{CGAL_Point_2<R> vertex(int i) const;}
       {returns the i'th vertex modulo 3  of~\var.}

\method{CGAL_Point_2<R> operator[](int i) const;}
       {returns \CCstyle{vertex(i)}.}


\method{bool is_degenerate() const;}
       {triangle \var\ is degenerate, if the vertices are collinear.}


\method{CGAL_Orientation orientation() const;}
       {returns the orientation of~\var.}

\method{CGAL_Oriented_side oriented_side(const CGAL_Point_2<R> &p) const;}
       {returns \CCstyle{CGAL_POSITIVE_SIDE}, 
	\CCstyle{CGAL_ON_ORIENTED_BOUNDARY} or 
        \CCstyle{CGAL_ON_NEGATIVE_SIDE}, depending on where point $p$ is.}

\method{CGAL_Bounded_side bounded_side(const CGAL_Point_2<R> &p) const;}
       {}


For convenience we provide the following boolean functions:

\method{bool has_on_positive_side(const CGAL_Point_2<R> &p) const;}
       {}

\method{bool has_on_negative_side(const CGAL_Point_2<R> &p) const;}
       {}

\method{bool has_on_boundary(const CGAL_Point_2<R> &p) const;}
       {}

\method{bool has_on_bounded_side(const CGAL_Point_2<R> &p) const;}
       {} 

\method{bool has_on_unbounded_side(const CGAL_Point_2<R> &p) const;}
       {} 


\method{CGAL_Triangle_2<R> opposite();}
       {returns a triangle where the boundary is oriented the other
        way round (this flips the positive and the negative side, but
        not the bounded and unbounded side).}

\method{CGAL_Bbox_2 bbox() const;}
       {returns a bounding box containing \var.}

\method{CGAL_Triangle_2<R>  transform(const CGAL_Aff_transformation_2<R> &at) const;}
       {returns the triangle obtained by applying $at$ on the three
        vertices of \var.}


\implementation
A triangle is internally represented as a triple of points. 

\end {classtemplate} 

% $Log: Triangle_2.tex,v $
% Revision 1.3  1996/07/08 12:19:15  fabri
% *** empty log message ***
%
% Revision 1.2  1996/03/13 15:42:07  fabri
% *** empty log message ***
%
% Revision 1.1  1995/10/19 18:22:12  fabri
% Initial revision
%

