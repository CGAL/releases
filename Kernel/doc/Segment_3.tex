% $Id: Segment_3.tex,v 1.2 1996/07/08 12:19:15 fabri Exp fabri $

\begin {classtemplate} {CGAL_Segment_3<R>}
\CCsection{3D Segment}

\definition  An object $s$ of the data type \classname\ is a directed
straight line segment in the three-dimensional space $\E_3$, i.e.\ a
straight line segment $[p,q]$ connecting two points $p,q \in
\R^3$. The segment is topologically closed, i.e.\  the end
points belong to it. Point $p$ is called the {\em source} and $q$
is called the {\em target} of $s$. The length of $s$ is the
Euclidean distance between $p$ and $q$. Note that there is only a function
to compute the square of the length, because otherwise we had to
perform a square root operation which is not defined for all
number types, which is expensive and not exact.

\creation
\creationvariable{s}

\CCstyle{#include <CGAL/Segment_3.h>}

\hidden \constructor{CGAL_Segment_3();}
             {introduces an uninitialized variable \var.}

\hidden \constructor{CGAL_Segment_3(const CGAL_Segment_3<R> &q);}
 	    {copy constructor.}

\constructor{CGAL_Segment_3(const CGAL_Point_3<R> &p, const CGAL_Point_3<R> &q);}
            {introduces a segment \var\ with source $p$
             and target $q$. It is directed from the source towards
             the target.}


\operations
\threecolumns{5cm}{4cm}

\hidden \method{CGAL_Segment_3<R> & operator=(const CGAL_Segment_3<R> &q);}
        {Assignment.}

\method{bool operator==(const CGAL_Segment_3<R> &q) const;}
       {Test for equality: Two segments are equal, iff their sources and
        targets are equal.}

\method{bool operator!=(const CGAL_Segment_3<R> &q) const;}
       {Test for inequality.}



\method{CGAL_Point_3<R> source() const;}
       {returns the source  of \var.}

\method{CGAL_Point_3<R> target() const;}
       {returns the target of \var.}

\method{CGAL_Point_3<R> min() const;}
       {returns the point of \var with smallest coordinate (lexicographically).}

\method{CGAL_Point_3<R> max() const;}
       {returns the point of \var with largest coordinate (lexicographically).}


\method{CGAL_Point_3<R> vertex(int i) const;}
       {returns source or target of \var:   \CCstyle{vertex(0)} returns
        the source, \CCstyle{vertex(1)} returns the target. 
        The parameter \CCstyle{i} is taken modulo 2, which gives 
        easy access to the other vertex.}

\method{CGAL_Point_3<R> point(int i) const;}
       {returns \CCstyle{vertex(i)}.}

\method{CGAL_Point_3<R> operator[](int i) const;}
       {returns \CCstyle{vertex(i)}.}

\method{R::FT squared_length() const;}
       {returns the squared length of \var. }

\method{CGAL_Direction_3<R> direction() const;}
       {returns the direction from source to target.}


\method{CGAL_Segment_3<R> opposite() const; }
       {returns a segment with source and target interchanged.}

\method{CGAL_Line_3<R> supporting_line() const;}
       {returns the line $l$ passing through \var. Line $l$  has the
        same orientation as segment \var, that is 
        from the source to the target of \var.}

\method{bool is_degenerate() const;}
       {segment \var\ is degenerate, if source and target fall together.}


\method{bool has_on(const CGAL_Point_3<R> &p) const;}
       {A point is on \var, iff it is equal to the source or target
        of \var, or if it is in the interior of \var.}
% 
% \method{bool collinear_has_on(const CGAL_Point_3<R> &p) const;}
%        {checks if point $p$ is on segment~\var. This function is faster
%         than function \CCstyle{has_on()}.
%         \precond $p$ is collinear to \var.}
% 
\method{CGAL_Bbox_3 bbox() const;}
       {returns a bounding box containing~\var.}

\method{CGAL_Segment_3<R>  transform(const CGAL_Aff_transformation_3<R> &t) const;}
       {returns the segment obtained by applying $t$ on the source
        and the target of \var.}

\implementation
A segment is internally represented by two points. 


\end{classtemplate} 

% $Log: Segment_3.tex,v $
% Revision 1.2  1996/07/08 12:19:15  fabri
% *** empty log message ***
%
% Revision 1.1  1996/03/13 15:42:07  fabri
% Initial revision
%
% Revision 1.1  1995/10/19 18:22:12  fabri
% Initial revision
%
