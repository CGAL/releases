
\chapter{Number Types}\label{Numbertype}

\cgal\ representation classes are parameterized by number types.  
Choosing the underlying arithmetic by choosing a number type
can be used to do exact geometric computation efficiently by
selecting number type and representation class appropriately,
depending on the problem and the input data that have to be handled,
and to provide a trade-off between efficiency and accuracy.  

In homogeneous representation, two number types are involved,
although only one of them appears as a template parameter in
the homogeneous representation class.
This type, for sake of simplicity and readability called ring type is
used for the representation of homogeneous coordinates and all 
internal computations. These internal computations are basically
division-free; rarely divisions are made if it is assured that second
operand divides the first one.
The ring type is a placeholder for an integer type (or an integral 
domain type) rather than for elements of arbitrary rings. 
The name should remember you that the division operation is not
really needed (with the exception mentioned above) for this number type.
Of course, more general number types can be used as a ring type 
in a homogeneous representation class as well.
In some computations, e.g.\ accessing Cartesian coordinates, divisions
cannot be avoided. In these computations a second number type, the
field type, is used. \cgal\ automatically generates this number
type as a \CCstyle{CGAL_Quotient}\LatexHtml{, cf.\ Subsection~\ref{CGAL_Quotient}}{}.
For the Cartesian representation there is only one number type that is
used for all calculations.

The representation classes provide access to the numbertypes 
involved in the representation, although it is not expected that
such access is needed at this level, since low-level geometric 
operations are wrapped in geometric primitives provided by the kernel.
This access can be useful if appropriate kernel primitives are 
missing.
In a homogeneous representation class \CCstyle{R} ring type and field
type can be accessed as \CCstyle{R::RT} and \CCstyle{R::FT}, respectively.
The number type used in Cartesian representation is considered as 
ring type and as field type depending on the context.
If can be accessed as \CCstyle{R::RT} and \CCstyle{R::FT}, according
to the use of number types used in the homogeneous counterpart.

\section{Required Functionality of Number Types\label{nt-requirements}}
Number types must fulfill certain requirements, such that they can
be successfully used in \cgal\ code.
This section describes those requirements.
We focus on the syntactical requirements. Of course, number types also
have evident semantic constraints. They should be meaningful in the
sense that they approximate the integers or the rationals 
or some other subfield of the real numbers.

The requirements are described as a class interface of a class
\CCstyle{NT}, with constructors, methods and the like. This is only a
matter of presentation.  In fact
{\tt double} and {\tt float} also fulfill the requirements.

\begin{class} {NT}
\creation
\creationvariable{ntvar}

\constructor{NT();}
	    {declaration of a variable.}


\constructor{NT(const NT &ntval);}
	    {declaration and initialization.}


\constructor{NT(i)}
	    {declaration and initialization with a small integer
constant $i$, $0 \leq i \leq 127$. The neutral elements for addition
(zero) and multiplication (one) are needed quite often, but sometimes
other small constants are useful too. The value 127 was chosen such
that even signed 8 bit number types can fulfill this condition.  }

\operations
\threecolumns{2cm}{4cm}
\method{NT & operator=(const NT &ntval);}
        {Assignment. 
% Note that we put no requirement on the return
% value of an assignemnt.  Hence, templated \cgal\ code must
% never contain \CCstyle{ntvar2 = ntvar1 = ntval}. 
}


\function{bool CGAL_is_valid(const NT &ntval);}
{Not all values of a number type need be valid. The routine
\CCstyle{CGAL_is_valid} checks this. For example, an expression like
\CCstyle{nt(0)/nt(0)} can result in an invalid number. Routines often
have as a precondition that all numerical values are valid.}

\function{bool CGAL_is_finite(const NT &ntval);}
{ When two large doubles are multiplied, the result may not fit in a
  \CCstyle{NT}. Some number types (the standard float and double type
  when they conform to standards) have a way to represent a too big
  value as infinity.  \CCstyle{CGAL_is_finite} implies
  \CCstyle{CGAL_is_valid}.}

\method{bool operator==(const NT &n) const;}
       {}

\method{bool operator!=(const NT &n) const;}
       {}

\method{bool operator<(const NT &n) const;}
       {}


\method{bool operator>(const NT &n) const;}
       {}


\method{bool operator<=(const NT &n) const;}
       {}


\method{bool operator>=(const NT &n) const;}
       {}

\function{NT operator+(const NT &ntval1, const NT &ntval2);}
       {}

\function{NT operator-(const NT &ntval1, const NT &ntval2);}
       {}

\function{NT operator*(const NT &ntval1, const NT &ntval2);}
       {}

\function{NT operator-(const NT &ntval);}
       {}

\function{double CGAL_to_double(const NT &ntval);}
         {gives the double value for a number type.
          This is usually an approximation for the real (stored) value.
          It can be used to send numbers to a renderer or to store them 
          in a file.}


\function{NT operator/(const NT &ntval1, const NT &ntval2);}
       {%Division is not required for a ring type.
Division by zero need not be defined. It may result in a runtime error, an
invalid value, a valid value or anything else. This basically means that the
library tests for zero whenever it does a division.}

\section{Utility Routines}

The previous section listed all the required functionality.
For the user of a number type it is handy to have a bigger set of
operations available.



\function{NT CGAL_min(const NT& ntval1, const NT& ntval2);}
{returns the smaller of the two values.}

\function{NT CGAL_max(const NT& ntval1, const NT& ntval2);}
{returns the larger of the two values.}

\function{NT CGAL_abs(const NT& ntval);}
{the absolute value.}


\function{int CGAL_sign(const NT& ntval);}
{the sign: -1, 0, or $+1$.}

\function{bool CGAL_is_negative(const NT& ntval);}
{}

\function{bool CGAL_is_positive(const NT& ntval);}
{}

\function{bool CGAL_is_zero(const NT& ntval);}
{}

\function{bool CGAL_is_one(const NT& ntval);}
{}

Those routines are implemented using the required operations from the
previous section. They are defined by means of templates, so you do not
have to supply all those operations when you write  a new number type.

\end{class} 

\section{Built-in Number Types}

%GJG This does not say very clearly that double and float can be used as 
%GJG valid number types. Nor does it say which header file should be included
%GJG for this to be so. I added a paragraph.

%AF I wanted to include these files by default

The built in number types {\tt float} and {\tt double} have the
required arithmetic and comparison operators. They lack some required
routines though which are automatically included by \cgal.
\LatexHtml{\footnote{ The functions can be found in the header files 
{\tt CGAL/Double.h} and {\tt CGAL/Float.h}.}}{}


All built-in number types of \CC\ can represent a discrete (bounded)
subset of the rational numbers only.  We assume that the
floating-point arithmetic of your machine follows {\sc Ieee}
floating-point-standard.  Since the floating-point culture has much
more infrastructural support (hardware, language definition and
compiler) than exact computation it is very efficient.
Like with all number types with finite precision representation
which are used as approximations to the infinite ranges of 
integers or real numbers the built-in number types are inherently
potentially inexact.
Be aware of this if you decide to use the efficient built-in 
number types:
You have to cope with numerical problems.  For example, you can
compute the intersection point of two lines and then check whether
this point lies on the two lines. 
%With exact arithmetic, the check will always return {\tt true}. 
With floating point arithmetic,
roundoff errors may cause the answer of the check to be {\tt false}. 
With the built-in integer types overflow might occur.

\section{Number Types Provided by LEDA}

\leda\ provides number types that can be used for exact computation 
with both Cartesian and homogeneous representation.  If you are using
homogeneous representation with the built-in integer types {\tt
short}, {\tt int}, and {\tt long} as ring type, exactness of
computations can be guaranteed only if your input data come from a
sufficiently small integral range and the depth of the computations is
sufficiently small.  \leda\ provides the number type {\tt integer} for
integers\LatexHtml{\footnote{Note the subtle difference between integer and
{\tt integer}. Whereas the first denotes \Z, the second denotes a
class in \leda.}}{}, of arbitrary length. (Of course the length is
somehow bounded by the resources of your computer.)  It can be used as
ring type in homogeneous representation and leads to exact
computation as long as all intermediate results are rational.  For the
same kind of problems Cartesian representation with number type {\tt
rational} leads to exact computation as well.

The most sophisticated number type in \leda\ is the number type called
{\tt real}. Like in Pascal, where the name {\tt real} is used for
floating-point numbers, the name {\tt real} does not describe the
number type exactly.  {\tt reals} are a subset of real algebraic
numbers.  Any integer is {\tt real} and {\tt real}s are closed under
the operations $+,-,*,/$ and $\sqrt{\ }$. {\tt real}s guarantee that
all comparisons between expressions involving {\tt real}s produce the
exact result.

In the files {\tt CGAL/Integer.h}, {\tt CGAL/Rational.h}, and
{\tt CGAL/Real.h}, the \leda\ types {\tt integer}, {\tt rational},
and {\tt real} are made conform to the requirements presented in
\Section{nt-requirements}. 
Also, in these files the \leda\ number types are included.
For more details on the number types of \leda\ we refer to the \leda\ 
manual~\cite{leda-manual}.


\section{Number Types Provided by CGAL}

Currently, \cgal\ has no proper number types. 
%% 
%% In the future
%% \cgal\ will provide a number type {\tt CGAL\_Integral\_double} 
%% that stores integer values in variables of the type {\tt double}.  
%% This has three advantages. First, according to {\sc Ieee} 
%% standard the type {\tt
%% double} uses 53 bits for the mantissa of the number (one of them
%% implicitly by the normalization assumption), 1 bit for the sign, and
%% 11 bits for the exponent.  This allows to store integers between
%% $-2^{53}$ and $2^{53}$ which gives us a larger range than what you can
%% get with the built-in type {\tt int}, which uses 32 bit on most platforms.
%% 
%% 
%% Secondly, most computers have floating-point units for fast {\tt
%% double} arithmetic. The gain in speed carries through to integers
%% stored in double variables.  Some machines even do not have special
%% hardware for integers anymore.  The floating-point unit is used e.g.\
%% for integer multiplication anyhow.
%% 
%% Finally, if a result of an arithmetic operation exceeds the range
%% $[-2^{53},2^{53}]$, the 
%% {\tt CGAL\_Integral\_double}s overflow nicely, the
%% result approximates the correct value.  This is not true in the {\tt
%% int} arithmetic.  On the other hand computations with {\tt int} are
%% modulo $2^{32}$, which also has some nice effects.  If you have two
%% expressions over the integers involving $+,-,*$ only, that give the
%% same result, a test will report equality even if an overflow
%% occurs. However the test falsely reports equality if the results
%% differ by $2^{32}$.
%% 
%% The main advantage however is, that fast computation of the sign of
%% the determinant of a matrix with entries of type 
%% {\tt CGAL\_Integral\_double}
%% according to \cite{ABDPY} is used in the geometric predicates of
%% \cgal.
%% 
%% There will also be a number type {\tt CGAL\_Filtered\_integer}.  If this
%% number type is used (with homogeneous representation) floating-point
%% filters make exact computation in the geometric predicates of \cgal\
%% more efficient.
%% 

\clearpage
\begin {classtemplate} {CGAL_Quotient<NT>}
\label{CGAL_Quotient}
\CCsection{Quotient}

\definition
An object of the class \CCstyle{CGAL_Quotient<NT>} is an element of the 
ring of quotients of the integral domain type \CCstyle{NT}.
If \CCstyle{NT} behaves like an integer, \CCstyle{CGAL_Quotient<NT>}
behaves like the rational numbers. \leda's class
\CCstyle{rational} has been the basis for \CCstyle{CGAL_Quotient<NT>}.
A \CCstyle{CGAL_Quotient<NT>} \CCstyle{q} is represented as a pair of 
\CCstyle{NT}s, representing numerator and denominator.


\creation
\creationvariable{q}

\constructor{CGAL_Quotient();}
             {introduces an uninitialized variable \var.}

\hidden \constructor{CGAL_Quotient(const CGAL_Quotient<NT> &q);}
 	    {copy constructor.}

\constructor{CGAL_Quotient(const NT& n)}
            {introduces the quotient \CCstyle{n/1}.}

\constructor{CGAL_Quotient(const NT& n, const NT& d)}
            {introduces the quotient \CCstyle{n/d}.}


\operations
The arithmetic operations $+,\ -,\ *,\ /,\ +=,\
-=,\ *=,\ /=,\ -$(unary),
the comparison operations $<,\ <=,\ >,\ 
>=,\ ==,\ !=$ and the stream operations are all available.

\threecolumns{5cm}{4cm}

There are two access functions, namely to the
numerator and the denominator of a quotient.
Note that these values are not uniquely defined. 
It is guaranteed that \CCstyle{q.numerator()} and 
\CCstyle{q.denominator()} return values \CCstyle{nt_num} and
\CCstyle{nt_den} such that \CCstyle{q = nt_num/nt_den}, only
if  \CCstyle{q.numerator()} and \CCstyle{q.denominator()} are called
consequtively wrt \CCstyle{q}, i.e.~\CCstyle{q} is not involved in 
any other operation between these calls.

\method{NT numerator() const;}
       {returns a numerator of \CCstyle{q}.}

\method{NT denominator() const;}
       {returns a denominator of \CCstyle{q}.}

\hidden \method{CGAL_Quotient<NT>& normalize();}
{}

The following functions are added to fulfill the \cgal\ requirements
on number types.

\function{double CGAL_to_double(const CGAL_Quotient<NT>& q);}
       {returns some double approximation to \CCstyle{q}.}

\function{bool  CGAL_is_valid(const CGAL_Quotient<NT>& q);}
       {returns true, if numerator and denominator are valid.}

\function{bool  CGAL_is_finite(const CGAL_Quotient<NT>& q);}
       {returns true, if numerator and denominator are finite.}

\end {classtemplate} 



\section{User-supplied Number Types}

You can also use your own number type with the \cgal\ representation
classes, e.g.\  the {\sc BigNum} package \cite{BigNum}, or classes
built on top of the {\sc Gnu} Multiple  Precision Library~\cite{Gmp}.
Depending on the arithmetic operations carried out by the algorithms
that you are going to use the number types must fulfill the
requirements from \Section{nt-requirements}.  The \cgal\ support library 
provides a framework to check the compliance of  number-types. 
