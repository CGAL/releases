% $Id: Tetrahedron_3.tex,v 1.1 1996/03/13 15:42:07 fabri Exp fabri $


\begin {classtemplate} {CGAL_Tetrahedron_3<R>}
\CCsection{3D Tetrahedron}

\definition  An object $t$ of the class \classname\ is an oriented
tetrahedron in the three-dimensional Euclidean space $\E_3$. 

It is defined by four vertices $p_0$, $p_1$, $p_2$ and $p_3$.
The orientation of a tetrahedron is the orientation of its four 
vertices. That means it is positive when $p_3$ is on the positive
side of the plane defined by $p_0$, $p_1$ and $p_2$.

The tetrahedron itself splits the space $E_3$ in a {\em positive} and
a {\em negative} side.\LatexHtml{\footnote{See Section~\ref{Predicates} 
for the definition of \CCstyle{CGAL_Oriented_side}.}}{}
 
The boundary of a tetrahedron splits the space in two open regions, a
bounded one and an unbounded one.

\creation
\creationvariable{t}

\CCstyle{#include <CGAL/Tetrahedron_3.h>}

\hidden \constructor{CGAL_Tetrahedron_3();}
             {introduces an uninitialized variable \var.}

\hidden \constructor{CGAL_Tetrahedron_3(const CGAL_Tetrahedron_3<R> &u);}
 	    {copy constructor.}


\def\CCalternateThreeColumn{\CCtrue}
\constructor{CGAL_Tetrahedron_3(const CGAL_Point_3<R> &p0, 
	                     const CGAL_Point_3<R> &p1, 
	                     const CGAL_Point_3<R> &p2, 
	                     const CGAL_Point_3<R> &p3);}
            {introduces a tetrahedron \var\ with vertices $p_0$, $p_1$, $p_2$ and $p_3$.}


\operations
\threecolumns{5cm}{4cm}

\hidden \method{CGAL_Tetrahedron_3<R> & operator=(const CGAL_Tetrahedron_3<R> &t2);}
        {Assignment.}

\method{bool operator==(const CGAL_Tetrahedron_3<R> &t2) const;}
       {Test for equality: two tetrahedra are equal, iff there exists a 
        cyclic permutation of the vertices of $t2$, such that they are 
        equal to the vertices of~\var.}

\method{bool operator!=(const CGAL_Tetrahedron_3<R> &t2) const;}
       {Test for inequality.}

\method{CGAL_Point_3<R> vertex(int i) const;}
       {returns the i'th vertex modulo 4  of~\var.}

\method{CGAL_Point_3<R> operator[](int i) const;}
       {returns \CCstyle{vertex(int i)}.}


\method{bool is_degenerate() const;}
       {Tetrahedron \var\ is degenerate, if the vertices are coplanar.}


\method{CGAL_Orientation    orientation() const;}
       { }

\method{CGAL_Oriented_side  oriented_side(const CGAL_Point_3<R> &p) const;}
       {}

\method{CGAL_Bounded_side  bounded_side(const CGAL_Point_3<R> &p) const;}
       {}

For convenience we provide the following boolean functions:

\method{bool has_on_positive_side(const CGAL_Point_3<R> &p) const;}
       {}

\method{bool has_on_negative_side(const CGAL_Point_3<R> &p) const;}
       {}


\method{bool has_on_boundary(const CGAL_Point_3<R> &p) const;}
       {}

\method{bool has_on_bounded_side(const CGAL_Point_3<R> &p) const;}
       {}

\method{bool has_on_unbounded_side(const CGAL_Point_3<R> &p) const;}
       {}


\method{CGAL_Bbox_3 bbox() const;}
       {returns a bounding box containing \var.}

\method{CGAL_Tetrahedron_3<R>  transform(const CGAL_Aff_transformation_3<R> &at) const;}
       {returns the tetrahedron obtained by applying $at$ on the three
        vertices of~\var.}



\end {classtemplate} 

% $Log: Tetrahedron_3.tex,v $
% Revision 1.1  1996/03/13 15:42:07  fabri
% Initial revision
%
