% $Id: Triangle_3.tex,v 1.1 1996/03/13 15:42:07 fabri Exp fabri $


\begin {classtemplate} {CGAL_Triangle_3<R>}
\CCsection{3D Triangle}

\definition  An object $t$ of the class \classname\ is a triangle in
the three-dimensional Euclidean space $\E_3$. As the triangle is not
a full-dimensional object there is only a test whether a point lies on
the triangle or not.
 
\creation
\creationvariable{t}

\CCstyle{#include <CGAL/Triangle_3.h>}

\hidden \constructor{CGAL_Triangle_3();}
             {introduces an uninitialized variable \var.}

\hidden \constructor{CGAL_Triangle_3(const CGAL_Triangle_3<R> &u);}
 	    {copy constructor.}


\def\CCalternateThreeColumn{\CCtrue}
\constructor{CGAL_Triangle_3(const CGAL_Point_3<R> &p, 
	                     const CGAL_Point_3<R> &q, 
	                     const CGAL_Point_3<R> &r);}
            {introduces a triangle \var\ with vertices $p$, $q$ and $r$.}


\operations
\threecolumns{5cm}{4cm}

\hidden \method{CGAL_Triangle_3<R> & operator=(const CGAL_Triangle_3<R> &t2);}
        {Assignment.}

\method{bool operator==(const CGAL_Triangle_3<R> &t2) const;}
       {Test for equality: two triangles t and $t_2$ are equal, iff there 
        exists a cyclic permutation of the vertices of $t2$, such that 
        they are equal to the vertices of~\var.}

\method{bool operator!=(const CGAL_Triangle_3<R> &t2) const;}
       {Test for inequality.}

\method{CGAL_Point_3<R> vertex(int i) const;}
       {returns the i'th vertex modulo 3  of~\var.}

\method{CGAL_Point_3<R> operator[](int i) const;}
       {returns \CCstyle{vertex(int i)}.}


\method{bool is_degenerate() const;}
       {Triangle \var\ is degenerate, if the vertices are collinear.}


\method{bool has_on(const CGAL_Point_3<R> &p) const;}
       {A point is on \var, if it is on a vertex, an edge or the
        face of \var.}

\method{CGAL_Bbox_3 bbox() const;}
       {returns a bounding box containing \var.}

\method{CGAL_Triangle_3<R>  transform(const CGAL_Aff_transformation_3<R> &at) const;}
       {returns the triangle obtained by applying $at$ on the three
        vertices of~\var.}



\end {classtemplate} 

% $Log: Triangle_3.tex,v $
% Revision 1.1  1996/03/13 15:42:07  fabri
% Initial revision
%
