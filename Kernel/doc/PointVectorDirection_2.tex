
\cleardoublepage
\chapter{2D Point, Vector and Direction \label{PointVectorDirection}}


We strictly distinguish between points, vectors and directions: 
A {\em point} is a point in the two-dimensional euclidean plane
$\E_2$, a {\em vector} is the difference of two points $p_2$, $p_1$
and denotes the direction and the distance from $p_1$ to $p_2$ in the
vector space $\R^2$, a {\em direction} is a vector where we forget
about its length.

They are different mathematical concepts, for example they behave
different under affine transformations and an addition of two
points is meaningless in affine geometry.  By putting them in different
classes we not only get cleaner code, but also type checking from the
compiler which avoids ambigous expressions. Hence, it pays twice to
make this distinction.

% $Id: Point_2.tex,v 1.3 1996/07/08 12:19:15 fabri Exp fabri $

\begin {classtemplate} {CGAL_Point_2<R>}
\CCsection{2D Point}

\definition
An object of the class \classname\ is a point in the two-dimensional
euclidean plane $\E_2$. 


%% \cgal\ defines a symbolic constant \CCstyle{CGAL_ORIGIN}  which denotes
%% the point at the origin. It can be used wherever a point can be used,
%% with the only exception that you can not access its dimension as it is
%% dimensionless.

Remember that \CCstyle{R::RT} and \CCstyle{R::FT} denote a ring type
and a field type. For the representation class
\CCstyle{CGAL_Cartesian<T>} the two types are equivalent. For the
representation class \CCstyle{CGAL_Homogeneous<T>} the ring type is
\CCstyle{R::RT} == \CCstyle{T} and the field type is \CCstyle{R::FT} == 
\CCstyle{CGAL_Quotient<T>}.

\creation
\creationvariable{p}

\CCstyle{#include <CGAL/Point_2.h>}

\hidden \constructor{CGAL_Point_2();}
             {introduces an uninitialized variable \var.}

\hidden \constructor{CGAL_Point_2(const CGAL_Point_2<R> &q);}
 	    {copy constructor.}

\constructor{CGAL_Point_2(const R::RT &hx, const R::RT &hy, const R::RT &hw = R::RT(1));}
            {introduces a point \var\ initialized to $(hx/hw,hy/hw)$.
             If the third argument is not explicitely given it defaults
             to \CCstyle{R::RT(1)}.}


\operations
\threecolumns{5cm}{4cm}

\hidden \method{CGAL_Point_2<R> & operator=(const CGAL_Point_2<R> &q);}
        {Assignment.}

\method{bool operator==(const CGAL_Point_2<R> &q) const;}
       {Test for equality: Two points are equal, iff their $x$ and $y$ 
        coodinates are equal.}

\method{bool operator!=(const CGAL_Point_2<R> &q) const;}
       {Test for inequality.}


There are two sets of coordinate access functions, namely to the
homogeneous and to the Cartesian coordinates. They can be used
independently from the chosen representation type \CCstyle{R}.

\method{R::RT hx() const;}
       {returns the homogeneous $x$ coordinate.}

\method{R::RT hy() const;}
       {returns the homogeneous $y$ coordinate.}

\method{R::RT hw() const;}
       {returns the homogenizing  coordinate.}

Here come the Cartesian access functions. Note that you do not loose
information with the homogeneous representation, because then the field
type is a quotient.

\method{R::FT x() const;}
       {returns the Cartesian $x$ coordinate, that is $hx/hw$.}

\method{R::FT y() const;}
       {returns the Cartesian $y$ coordinate, that is $hy/hw$.}


The following operations are for convenience and for making this
point class compatible with code for higher dimensional points.
Again they come in a Cartesian and homogeneous flavor.

\method{R::RT homogeneous(int i) const;}
       {returns the i'th homogeneous coordinate of \var, starting with 0.
        \precond $0\leq i \leq 2$.}

\method{R::FT cartesian(int i) const;}
       {returns the i'th Cartesian coordinate of \var, starting with 0.
        \precond $0\leq i \leq 1$.}

\method{R::FT operator[](int i) const;}
       {returns \CCstyle{cartesian(i)}.
        \precond $0\leq i \leq 1$.}

\method{int dimension() const;}
       {returns the dimension (the constant 2).}

\method{CGAL_Bbox_2 bbox() const;}
       {returns a bounding box containing \var. Note that bounding boxes
        are not parameterized with whatsoever. }

\method{CGAL_Point_2<R>  transform(const CGAL_Aff_transformation_2<R> &t) const;}
       {returns the point obtained by applying $t$ on \var.}



The following operations can be applied on points:

\function{CGAL_Vector_2<R> operator-(const CGAL_Point_2<R> &p,
                                    const CGAL_Point_2<R> &q);}
       {returns the difference vector between \CCstyle{q} and \CCstyle{p}.}

\function{CGAL_Point_2<R> operator+(const CGAL_Point_2<R> &p,
                                    const CGAL_Vector_2<R> &v);}
       {returns a point obtained by translating \CCstyle{p} by the 
        vector \CCstyle{v}.}

\function{CGAL_Point_2<R> operator-(const CGAL_Point_2<R> &p,
                                    const CGAL_Vector_2<R> &v);}
       {returns a point obtained by translating \CCstyle{p} by the 
        vector \CCstyle{-v}.}

\example

The following declaration creates two points with Cartesian double coordinates.

\begin{cprog}

  CGAL_Point_2< CGAL_Cartesian<double> > p, q(1.0, 2.0);
\end{cprog} 

The variable {\tt p} is uninitialized and should first be used on 
the left hand side of an assignment. 
\begin{cprog}

  p = q;

  cout << p.x() << "  " << p.y() << endl; 
\end{cprog} 
\end {classtemplate} 

% $Log: Point_2.tex,v $
% Revision 1.3  1996/07/08 12:19:15  fabri
% *** empty log message ***
%
% Revision 1.2  1996/03/13 15:42:07  fabri
% *** empty log message ***
%
% Revision 1.1  1995/10/19 18:22:12  fabri
% Initial revision
%


\newpage
% $Id: Vector_2.tex,v 1.2 1996/07/08 12:19:15 fabri Exp fabri $

\begin {classtemplate} {CGAL_Vector_2<R>}
\CCsection{2D Vector}

\definition

An object of the class \classname\ is a vector in the two-dimensional 
vector space $\R^2$. Geometrically spoken a vector is the difference
of two points $p_2$, $p_1$ and denotes the direction and the distance
from   $p_1$ to $p_2$. 

\cgal\ defines a symbolic constant \CCstyle{CGAL_NULL_VECTOR}. We 
will explicitely state where you can pass this constant as an argument
instead of a vector initialized with zeros.


\creation
\creationvariable{v}

\CCstyle{#include <CGAL/Vector_2.h>}

\hidden\constructor{CGAL_Vector_2();}
             {introduces an uninitialized variable \var.}

\hidden \constructor{CGAL_Vector_2(const CGAL_Vector_2<R> &w);}
 	    {copy constructor.}

\constructor{CGAL_Vector_2(const R::RT &hx, const R::RT &hy, const R::RT &hw = R::RT(1));}
            {introduces a vector \var\ initialized to $(hx/hw,hy/hw)$.
             If the third argument is not explicitely given it defaults
             to \CCstyle{R::RT(1)}.}


\operations
\threecolumns{5cm}{4cm}

\hidden \method{CGAL_Vector_2<R> & operator=(const CGAL_Vector_2<R> &w);}
        {Assignment.}

\method{bool operator==(const CGAL_Vector_2<R> &w) const;}
       {Test for equality: two vectors are equal, iff their $x$ and $y$ 
        coodinates are equal. You can compare with the
        \CCstyle{CGAL_NULL_VECTOR}.}

\method{bool operator!=(const CGAL_Vector_2<R> &w) const;}
       {Test for inequality. You can compare with the
        \CCstyle{CGAL_NULL_VECTOR}.}


There are two sets of coordinate access functions, namely to the
homogeneous and to the Cartesian coordinates. They can be used
independently from the chosen representation type \CCstyle{R}.

\method{R::RT hx() const;}
       {returns the homogeneous $x$ coordinate.}

\method{R::RT hy() const;}
       {returns the homogeneous $y$ coordinate.}

\method{R::RT hw() const;}
       {returns the homogenizing  coordinate.}

Here come the Cartesian access functions.  Note that you do not loose
information with the homogeneous representation, because then the field
type is a quotient.


\method{R::FT x() const;}
       {returns the \CCstyle{x}-coordinate of \var, that is $hx/hw$.}

\method{R::FT y() const;}
       {returns the \CCstyle{y}-coordinate of \var, that is $hy/hw$.}

The following operations are for convenience and for making the
class \classname\ compatible with code for higher dimensional vectors.
Again they come in a Cartesian and homogeneous flavor.

\method{R::RT homogeneous(int i) const;}
       {returns the i'th homogeneous coordinate of \var, starting with 0.
        \precond $0\leq i \leq 2$.}

\method{R::FT cartesian(int i) const;}
       {returns the i'th Cartesian coordinate of \var, starting at 0.
        \precond $0\leq i \leq 1$.}

\method{R::FT operator[](int i) const;}
       {returns  \CCstyle{coordinate(i)}.
        \precond $0\leq i \leq 1$.}

\method{int dimension() const;}
       {returns the dimension (the constant 2).}

\method{CGAL_Vector_2<R>  transform(const CGAL_Aff_transformation_2<R> &t) const;}
       {returns the vector obtained by applying $t$ on \var.}

\method{CGAL_Vector_2<R>  perpendicular(const CGAL_Orientation &o) const;}
       {returns the vector perpendicular to \var\ in clockwise or
        counterclockwise orientation.}

The following operations can be applied on vectors:

\method{CGAL_Vector_2<R>        operator+(const CGAL_Vector_2<R> &w) const;}
       {Addition.}

\method{CGAL_Vector_2<R>        operator-(const CGAL_Vector_2<R> &w) const;}
       {Subtraction.}

\method{CGAL_Vector_2<R>        operator-() const;}
       {returns the opposite vector.}

\method{R::FT                  operator*(const CGAL_Vector_2<R> &w) const;}
       {returns the scalar product (= inner product) of the two vectors.}


\method{CGAL_Vector_2<R> operator*(const R::RT &s) const;}
       {Multiplication with a scalar from the right. Although it would
        be more natural, \cgal does not offer a multiplication with a 
        scalar from the left.\LatexHtml{\footnote{This is due to compiler problems
        which will hopefully be overcome sooner or later.}}{This is due to compiler problems
        which will hopefully be overcome sooner or later.}}


\method{CGAL_Vector_2<R> operator*(const R::FT &s) const;}
       {Multiplication with a scalar from the right.}


%\function{CGAL_Vector_2<R> operator*(const R::RT &s, 
%	                               const CGAL_Vector_2<R> &w);}
%       {Multiplication with a scalar from the left.}

\method{CGAL_Vector_2<R>        operator/(const R::RT &s) const;}
       {Division by a scalar.}

\method{CGAL_Direction_2<R> direction() const;}
       {returns the direction which passes through \var.}



\end {classtemplate} 

% $Log: Vector_2.tex,v $
% Revision 1.2  1996/07/08 12:19:15  fabri
% *** empty log message ***
%
% Revision 1.1  1995/10/19 18:22:12  fabri
% Initial revision
%


\newpage
% $Id: Direction_2.tex,v 1.3 1996/07/08 12:19:15 fabri Exp fabri $

\begin {classtemplate} {CGAL_Direction_2<R>}
\CCsection{2D Direction}

\definition
An object of the class \classname\ is a vector in the two-dimensional 
vector space $\R^2$  where we forget about its length. They can be
viewed as unit vectors, although there is no normalization internally,
since this is error prone.  Directions are used whenever the length of
a vector does not matter, e.g.\ you can ask for the direction
orthogonal to an oriented plane, the direction of an oriented line,
and so on. They further can be used like angles. The slope of a direction
is \CCstyle{dy()/dx()}.

\creation
\creationvariable{d}

\CCstyle{#include <CGAL/Direction_2.h>}

\hidden \constructor{CGAL_Direction_2();}
             {introduces an uninitialized direction \var.}

\hidden \constructor{CGAL_Direction_2(const CGAL_Direction_2<R> &d);}
 	    {copy constructor.}

\constructor{CGAL_Direction_2(const CGAL_Vector_2<R> &v);}
	    {introduces the direction \var\ of vector $v$.}

\constructor{CGAL_Direction_2(const R::RT &x, const R::RT &y);}
            {introduces a direction \var\ passing through the point 
             at $(x, y)$.}


\operations
\threecolumns{5cm}{4cm}

\hidden \method{CGAL_Direction_2<R> & operator=(const CGAL_Direction_2<R> &e);}
        {Assignment.}

\method{R::RT delta(int i) const;}
       {returns the i'th value of the slope of \var.
        \precond: $0 \leq i \leq 1$.}

\method{R::RT dx() const;}
       {returns the $dx$ value of the slope of \var.}

\method{R::RT dy() const;}
       {returns the $dy$ value of the slope of \var.}

There is a total order on directions. We compare the angles between the
positive $x$-axis and the directions in counterclockwise direction.

\method{bool operator==(const CGAL_Direction_2<R> &e) const;}
       {Test for equality.}

\method{bool operator!=(const CGAL_Direction_2<R> &e) const;}
       {Test for inequality.}

\method{bool operator<(const CGAL_Direction_2<R> &e) const;}
       {}

\method{bool operator>(const CGAL_Direction_2<R> &e) const;}
       {}

\method{bool operator<=(const CGAL_Direction_2<R> &e) const;}
       {}

\method{bool operator>=(const CGAL_Direction_2<R> &e) const;}
       {}

\method{bool counterclockwise_in_between(const CGAL_Direction_2<R> &d1,
                                   const CGAL_Direction_2<R> &d2) const;}
       {}


\method{CGAL_Direction_2<R>  operator-() const;}
       {The direction opposite to \var.}

\method{CGAL_Vector_2<R> vector() const;}
       {returns a vector that has the same direction as \var.}

\method{CGAL_Direction_2<R>  transform(const CGAL_Aff_transformation_2<R> &t) const;}
       {returns the direction obtained by applying $t$ on \var.}


\end {classtemplate} 

% $Log: Direction_2.tex,v $
% Revision 1.3  1996/07/08 12:19:15  fabri
% *** empty log message ***
%
% Revision 1.2  1996/03/13 15:42:07  fabri
% *** empty log message ***
%
% Revision 1.1  1995/10/19 18:22:12  fabri
% Initial revision
%



\section{Conversion between Points and Vectors\label{conversion}}

We said that it does not make sense to add two points, but it does
make sense to subtract them and the result should be a vector. 
\cgal\ defines a symbolic constant \CCstyle{CGAL_ORIGIN}  which denotes
the point at the origin.  Subtracting it from a
point $p$ results in the locus vector of $p$. 

\begin{cprog}

  CGAL_Point_2< CGAL_Cartesian<double> >  p(1.0, 1.0), q;

  CGAL_Vector2< CGAL_Cartesian<double> >  v;

  v = p - CGAL_ORIGIN;

  q = CGAL_ORIGIN + v;  
\end{cprog} 

In order to obtain the point corresponding to a vector $v$ you simply
have to add $v$ to \CCstyle{CGAL_ORIGIN}. If you want to determine 
the point $q$ in the middle between two points $p_1$ and $p_2$, you can write

\begin{cprog}

  q = p_1 + (p_2 - p_1) / 2.0;
\end{cprog}  



\section{Implementation}

Points, vectors and directions use a handle/representative mechanism:
A handle is an intelligent pointer, a representative is an object with a
reference counter. The three classes have the same internal
representation, namely a tuple of coordinates (plus a homogenizing
coordinate in the case of homogeneous coordinates), which makes
assignment, copy constructors and type conversion cheap. 

An assigment makes the handle of the left hand side point to the
representative the handle on the right hand sidepoints to.  The copy
constructor creates a new handle which points to the same
representative. This especially pays if your coordinates are of
non-constant size.

What about conversion? We explained that you convert by subtracting
the origin from a point to obtain its locus vector, or by adding a
vector to the origin to obtain the corresponding point.  Although the
origin behaves like a point,  \CCstyle{CGAL_ORIGIN} is not an object of
the class \CCstyle{CGAL_Point_2<R>} but of the class
\CCstyle{CGAL_Origin}.  All constructors and operators taking a point
as argument are overloaded with the origin class in order to avoid
memory allocation (for a point with coordinates zero), and arithmetic 
operations (where numbers would be added to zero). To give an example:
When you ``add'' a vector $v$ to the origin only a new handle is created
which points to the representation of $v$.




